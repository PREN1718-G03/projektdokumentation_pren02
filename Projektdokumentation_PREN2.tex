\documentclass[a4paper]{report}

% Use swiss german letters
\usepackage[utf8]{inputenc}

% Language: german
\usepackage[ngerman]{babel}

% Fancy Figures
\usepackage{graphicx}

\usepackage{subcaption}

% Colored text
\usepackage{xcolor}

% Subfigures
\usepackage{subcaption}

% Use Times
\usepackage{mathptmx}

% Display the Bibliography in the TOC
\usepackage{tocbibind}

% Better lists
\usepackage{enumitem}

% We want SI units!!!
\usepackage{siunitx}

% Formeln
\usepackage{amsfonts}

% Formeln
\usepackage{amsmath}

% Definierte Spaltenbreiten bei Tabellen
\usepackage{array}

% Pagebreaking tables
\usepackage{longtable}

% Use biblatex
\usepackage[style=apa,backend=biber,citestyle=authoryear]{biblatex}

% Footnote glue to bottom
\usepackage[bottom]{footmisc}

% Make references hyperlinks
\usepackage[hidelinks]{hyperref}

% To be able to use multiple columns
\usepackage{multicol}

% PDF einfügen
\usepackage{pdfpages}

\usepackage{pdflscape}

% Tell BibLatex to use the ngerman language mapping
\DeclareLanguageMapping{ngerman}{ngerman-apa}

% Define the bibliography file
\addbibresource{bibliography.bib}

% To let LaTeX handle "
\usepackage[autostyle=true, german=quotes]{csquotes}

% Rename the Abstract to Management Summary
\addto\captionsngerman{\renewcommand{\abstractname}{Management Summary}}

% Define new colors
\definecolor{grey}{HTML}{C5C5C5}
\definecolor{lightgrey}{HTML}{E6E6E6}
\definecolor{lightgray}{HTML}{F8F8F8}

% Titlepage
\newcommand*{\titleAP}{\begingroup % Create the command for including the title page in the document
	\centering
	\vspace*{\baselineskip} % Whitespace at the top of the page

	{Basil Bachmann, Pascal Baumann, Victor Guntern, Markus Kempf, Jan Odermatt, Simon Rohrer}\\[0.167\textheight] % Author name

	{\Huge\bfseries Projektdokumentation PREN Gruppe03}\\[\baselineskip]

	{\Large \textit{Autonome Laufkatze}}\\
	\today

	\vspace*{3\baselineskip} % Whitespace at the bottom of the page
	\endgroup}

% Define the path were images are found
\graphicspath{{./img/}{./PDF/}}

\begin{document}
\pagenumbering{gobble}

\titleAP

\newpage

\chapter*{Redlichkeitserklärung}
\label{ch*:Redlich}
Hiermit erklären wir, dass wir die vorliegende Arbeit selbständig angefertigt haben und keine anderen als die angegebenen Hilfsmittel verwendet wurden. Sämtliche verwendeten Textausschnitte, Zitate oder Inhalte anderer Verfasser wurden ausdrücklich als solche gekennzeichnet.

\vspace{1.5em}

\noindent
Horw, \today

\vspace{2em}

\noindent
\begin{tabular}{lp{0.7\textwidth}}
	Basil Bachmann & \includegraphics[height=1.5cm,keepaspectratio]{BasilBachmann}\\
	Pascal Baumann & \includegraphics[height=1.5cm,keepaspectratio]{PascalBaumann}\\
	Victor Guntern & \includegraphics[height=1.5cm,keepaspectratio]{VictorGuntern}\\
	Markus Kempf & \includegraphics[height=1.5cm,keepaspectratio]{MarkusKempf}\\
	Jan Odermatt & \includegraphics[height=1.5cm,keepaspectratio]{JanOdermatt}\\
	Simon Rohrer &  \includegraphics[height=1.5cm,keepaspectratio]{SimonRohrer}\\
\end{tabular}

\newpage

\pagenumbering{roman}
\begin{abstract}
	Im Rahmen des Moduls Produktentwicklung 2 wird ein Funktionsmuster des in PREN01 ausgearbeiteten Konzeptes erstellt. Dabei sind die Fachrichtungen Maschinentechnik, Informatik und Elektrotechnik vertreten. Neben der Realisierung des geplanten Funktionsumfang, steht vor allem auch das gute Abschneiden im Wettbewerb im Vordergrund.

	Der Fokus dieses Moduls liegt in dem Bauen der Laufkatze und der Funktionalität des ganzen Gerätes. Zudem ist die Projektplanung, die Dokumentation und die Interdisziplinäre Zusammenarbeit sehr wichtig. Das Ziel des Projektes ist es, eine Laufkatze zu entwickeln welche autonom eine Last aufheben und an einer zuvor unbekannten Position auf eine Zielplatte absetzten kann. Die Position der aufgenommenen Last soll während dem Transport angezeigt werden.

	Das Ergebnis dieser Arbeit ist ein Konzept einer Laufkatze, welche mit Seilrollen angetrieben wird. Unterhalb der Aufhängung ist die Montageplatte und die Hubeinrichtung angebracht. Diese besteht aus einem Hacken welcher mit einer geführten Seilwindenkonstruktion gehoben werden kann. Die Zielplatte wird mittels Bilderkennung erkannt und die Distanz mit Ultraschallsensoren und dem Zählen der Schritte von den Motoren.

	Am Ende des Moduls wird das geschaffene an einem Wettbewerb von Experten bewertet. Es werden unter anderem die Geschwindigkeit, die Genauigkeit und die fehlerfreie Absolvierung der Aufgaben angeschaut.

\end{abstract}

\chapter*{Versionierung}
\label{ch*:Vers}
\vspace{2em}

\noindent
\begin{tabular}{|c|p{0.7\textwidth}|}
	\hline
	\textbf{Versionsnummer} & \textbf{Beschreibung}\\
	\hline
	0.1 & Erstellung \\
	\hline
	0.2 & Testat 1\\
	\hline
	0.3 & Testat 2\\
	\hline
\end{tabular}

\tableofcontents

\newpage

\pagenumbering{arabic}

\chapter{Einleitung}
\label{ch:Intro}
Dieses Dokument beschreibt ein Funktionsmuster einer autonomen Laufkatze. Dabei handelt es sich um ein interdisziplinäres Projekt, welches über zwei Semester an der Hochschule Luzern absolviert wurde. In diesem Projekt musste mit einem begrenzten Budget gearbeitet und vielseitige Herausforderungen bewältigt werden.

In der vorhergegangenen Konzeptentwicklung konnte sichergestellt werden, dass alle Anforderungen durch unser Konzept erfüllt werden. Dieses Konzept gilt es nun auszuführen. Zu beginn wurde vor allem auf einen Prototyp hingearbeitet, welcher die Grundfunktionen ausführen konnte. Auf die Details und die autonomen Funktionen wurde erst später eingegangen.

Wie im vorherigen Modul Produktentwicklung 1 wurde die Dokumentation mit \LaTeX gestaltet, da es sich bewährte.

\section{Kurzbeschrieb Anforderungen}
\label{sec:KurzAnforder}
Das Gerät muss sich autonom entlang eines ansteigenden Drahtseils fortbewegen können. Eine Last muss an einem fix bestimmten Ort aufgehoben werden, ohne Berührung über nachfolgende Hindernisse transportiert und danach möglichst exakt auf einer Zielplatte im Absetzbereich abgesetzt werden (siehe Abb.\ref{fig:Funktionsskizze}).

Nach dem Absetzen der Last muss das Gerät den Zielraum erreichen und den Masten berühren. Das Gerät muss die Zielplatte selbstständig erkennen können. Das Gerät darf nur die Last, das Drahtseil und den zweiten Masten berühren. Nach Aufnahme der Last, muss die aktuelle Position der Last in x-, und z-Richtung in Echtzeit dargestellt werden.

Das Gerät muss innerhalb von zwei Minuten startbereit sein. Die Zeit um den Zielraum zu erreichen liegt bei maximal vier Minuten. Die Abbildung \ref{fig:Ablaufdiagramm} stellt den beschrieben Ablauf anschaulich dar.

In Abbildung \ref{fig:Blockdiagramm} findet sich eine Darstellung der technischen Komponenten und ihren Zusammenhängen.

\chapter{Produktbeschreibung}
% Beschreibung der Komponenten und des Produkts welche wir in PREN02 erarbeiten
\label{ch:Produktbeschreibung}

\section{Beschreibung der Grundfunktion}
\label{sec:GrundBeschrieb}
Die Grundfunktionen des zu bauenden Gerätes sind hauptsächlich das Fortbewegen an einem Drahtseil und das Greifen und Absetzen einer Last. Um die Last zu greifen ist ein Hubmechanismus notwendig, der die Last um mindestens 200mm heben kann. Um die aktuelle Position messen zu können, werden zwei Ultraschallsensoren verwendet. Die aktuelle Position der aufzuhebenden Last wird auf dem Gerät mittels eines LCD-Displays angezeigt. Eine Kamera wird verwendet, um die vorgegebene Zielplattform zu erkennen. Damit ist das Gerät in der Lage an der richtigen Position, über dem Mittelpunkt der Zielplattform, zu stoppen. Diese Funktionen laufen bis auf das Startsignal autonom ab.

\begin{figure}[h!]
	\includegraphics[keepaspectratio,width=\textwidth]{PrenFunktionsskizze}
	\caption{Übersichtsmodell}
	\label{fig:Funktionsskizze}
\end{figure}

\newpage
\section{Ablaufdiagramme}
\label{sec:Ablaufdiagramme}
Der Ablauf der verschiedenen Teilschritte erfolgt gemäss Abbildung \ref{fig:Ablaufdiagramm}.

\begin{figure}[h!]
	\includegraphics[keepaspectratio,width=\textwidth]{Ablaufdiagramm}
	\caption{Ablauf- und Zustandsdiagramm}
	\label{fig:Ablaufdiagramm}
\end{figure}

\newpage

Nach dem das Gerät am Seil angebracht wurde (\ref{sec:Bedienungsanleitung}), kann mit dem Restart-Knopf am \textit{ArduinoDue} (\ref{sssec:ArduinoDue}) gestartet werden. Es wird nun eine Vorprogrammierter Ablauf gefahren. Dieser beinhaltet den Antriebsschrittmotor zum Vorwärtsfahren (\ref{ssec:Antrieb}) bis der Haken in die Last eingehängt hat (\ref{ssec:Lastgreifenundabsetzen}) und den Hubschrittmotor, welcher die Last um eine bestimmte Schrittanzahl hebt (\ref{ssec:Lastanhebenundsenken}). Dieser Ablauf läuft über das \textit{ArduinoDue}. Gleichzeitig werden die Daten der Ultraschallsensoren (\ref{ssec:Ultraschallsensor}) über das \textit{RasPi} (\ref{sssec:RasPi}) ausgewertet und als Distanzen in Millimetern vom \textit{ArduinoDue} abgefragt. Bei der z-Koordinate wird zusätzlich noch die vom Hubschrittmotor gehobene Höhe abgezogen und beide Koordinaten werden auf dem LCD-Display (\ref{ssec:Positionsanzeige}) angezeigt.\\
Das Gerät wird dann mit der aufgenommenen Last Vorwärtsfahren, bis die Kamera (\ref{ssec:Zielerkennung}) die Zielplattform sieht und das \textit{RasPi} diese erkennen kann (\ref{ssec:Zielerkennung}). Das \textit{ArduinoDue} fragt regelmäßig ab, ob das \textit{RasPi} die Zielplattform bereits erkannt hat. Falls die Antwort ein positiver Wert ist, wird diese übermittelte Distanz abgefahren und der Hubmotor startet das Absetzen der Last. Wie viel der Hubmotor die Last absenken muss, wird aus den Daten des vertikal messenden Ultraschallsensors und den Schritten des geleisteten Hubmotors berechnet. Wenn die Last auf der Zielplattform aufkommt, klinkt sich der Haken automatisch aus der Last aus. Der Haken wird danach wieder 25 Zentimeter hochgehoben und der Antriebsmotor fährt solange vorwärts, bis der Endschalter(\ref{ssec:Endschalter}) betätigt wird. Damit ist die Aufgabe beendet.


\newpage
Im Blockdiagramm (Abb. \ref{fig:Blockdiagramm}) sind die Schnittstellen graphisch dargestellt.

\begin{figure}[h!]
	\centering
	\includegraphics[keepaspectratio,width=\textwidth]{Blockdiagramm}
	\caption{Blockdiagramm}
	\label{fig:Blockdiagramm}
\end{figure}

\newpage

\section{Komponenten}
\label{sec:Pren02Komponenten}

\begin{figure}[h!]
	\centering
	\includegraphics[height=0.5\textheight,keepaspectratio]{marku_103000_bg-00_ze1}
	\caption{Skizze des Lösungskonzeptes}
	\label{fig:LoesungsKonzept}
\end{figure}

\newpage

\subsection{Aufhängung}
%todo Abschnitt wahrscheinlich löschen
Durch die unten gelenkig und gedämpft gelagerte Aufhängung kann die Winkeländerung, welche durch den Seildurchhang verursacht wird, kompensiert werden. Dabei hängt die Last durch den Schwerpunkt immer senkrecht nach unten. Durch das Gelenk (ähnlich wie bei einer Gondelbahn) wird der ganze Aufbau aber empfindlich gegen pendeln, weil das Gelenk kein Drehmoment aufnehmen kann. Dieses wird beispielsweise durch die Beschleunigung bei Anfahren oder Abbremsen erzeugt. Das Gelenk wird mit einer Messingbuchse und einer Schraube mit Sicherungsmutter realisiert. An der Aufhängung wird der Hubmotor, die elektrischen Komponenten, die Energieversorgung und die Führungen der Hubeinrichtung angebracht. Diese besteht aus zwei Linearführungen, durch welche zwei Stäbe in einem Rohr hoch und hinuntergefahren werden.
\subsection{Antrieb}
Der Antrieb auf dem Seil wird mit drei Rädern ausgeführt. Das mittlere Rad aus Gummi ist das angetriebene Rad. Es wird direkt von einem Schrittmotor angetrieben. Das Antriebsrad wird über eine Hohlwelle direkt auf der Motorwelle befestigt. Dies setzt voraus, dass sich die Achse des Motors um den Radius des Antriebsrades über dem Seil befinden muss. Die Ausfallwahrscheinlichkeit ist dadurch tief, da keine Riemen-, oder Zahnradübersetzung notwendig sind. Zudem kann auch der Schlupf klein gehalten werden. Weil der Motor aber über dem Seil angebracht ist, liegt auch der Schwerpunkt des gesamten Gerätes höher. Der Lastausgleich wird dadurch schwieriger und Schwingungen werden begünstigt.
\\
Vor und hinter dem Antriebsrad sind im Abstand von jeweils 7cm zwei Führungsräder angebracht. Diese sind in der Höhe verstellbar und müssen manuell im voraus eingestellt werden. Die Aufgabe dieser Räder ist die Schwingung in x-Richtung zu minimieren und den Druck des Rades auf das Seil zu erhöhen. Letzteres hat zur Folge, dass die Reibkraft steigt und somit der Schlupf und die Rutschgefahr auf dem Seil minimiert wird.

\subsubsection{Schrittmotoren}
\label{sssec:Schrittmotoren}
%todo Beschreibung Schrittmotoren
Die Laufkatze hat total zwei Schrittmotoren verbaut. Einen für den Antrieb auf dem Seil und einen für den Hub der Last. Der Antrieb auf dem Seil ist ein Motor des Typs JK42HM48-1684A.Dieser dreht sich pro Schritt um \SI{0.9}{\degree}. Das Haltemoment beträgt 4.4kgcm. Der Antrieb des Hubes wird mit einem Motor des Typs 35HS3408A4 ausgeführt. Dieser dreht sich pro Schritt um \SI{1.8}{\degree}. Das Haltemoment beträgt 1.8kgcm.

\begin{figure}[h!]
	\centering
	\includegraphics[keepaspectratio,width=0.5\textwidth]{Antriebseinheit}
	\caption{Modulübersicht Antriebseinheit	}
	\label{fig:SystemArchitektur}
\end{figure}


\subsection{Aufhängung}
\label{ssec:Aufhaengung}
%todo Abschnitt wahrscheinlich löschen
Durch die unten gelenkige und gedämpft gelagerte Aufhängung kann die Winkeländerung, welche durch den Seildurchhang verursacht wird, kompensiert werden. Dabei hängt die Last durch den Schwerpunkt immer senkrecht nach unten. Durch das Gelenk (ähnlich wie bei einer Gondelbahn) wird der ganze Aufbau aber empfindlich gegen pendeln, weil das Gelenk kein Drehmoment aufnehmen kann. Dieses wird beispielsweise durch die Beschleunigung bei Anfahren oder Abbremsen erzeugt. Das Gelenk wird mit einer Messingbuchse und einer Schraube mit Sicherungsmutter realisiert. Zudem kann sich, je nach Aufbau, der Schwerpunkt der Laufkatze beim Anheben und Absetzen der Last ändern, was wiederum ein Drehmoment erzeugt. Mit Dämpfern kann das Drehmoment aufgenommen und ein Pendeln minimiert werden. Diese Dämpfer werden dann, im Prinzip einer Pendelstütze in der Mechanik, am einen Ende am Antrieb und am anderen Ende an der Aufhängung montiert. Zudem soll der Dämpfer möglichst horizontal angestellt sein, damit er die Pendelbewegung aufnehmen kann. AN der Aufhängung werden der Hubmotor, die elektrischen Komponenten, die Energieversorgung und die Führungen der Hubeinrichtung angebracht. Diese besteht aus zwei Linearführungen, durch welche zwei Stäbe in einem Rohr hoch und hinuntergefahren werden. Weiter werden, je nach Notwendigkeit, Ausgleichsgewichte auf der Aufhängung positioniert, damit die Laufkatze zentral unter dem Seil hängt.



\subsection{Grundeinheit}
\label{ssec:GrundeinheitBeschrieb}
Die Grundeinheit besteht aus der Verbindung von Hubvorrichtung und Fahreinheit. Dabei werden diverse elektronische Komponenten und Sensoren an der Grundeinheit befestigt, welche aber in einem separaten Kapitel behandelt werden. Auch zur Grundeinheit gehören die Motorhalterung, der Befestigungswickel für die Laufkatze und die Führungsrohre der Kohlenstofffaserstäbe.

\begin{figure}[h!]
	\includegraphics[keepaspectratio,width=\textwidth]{Grundplatte}
	\caption{Grundeinheit}
	\label{fig:Grundeinheit}
\end{figure}

\subsubsection{Grundplatte}
\label{sssec:GrundplatteBeschrieb}
Die Grundplatte wurde aus einer MFD-Platte gefertigt. Dabei hat diese Platte einen Ausschnitt in der Mitte, durch welches das Seil der Seilwinde geht.Neben dem Ausschnitt wird der Winkel für die Seilwinde angeschraubt. Zwei zusätzliche Löcher neben dem Ausschnitt dienen zur Befestigung der beiden Führungen. Die Elektronikkomponenten werden mithilfe von Klettverbindungen auf die Grundplatte befestigt. Die Halterung für den LCD-Display wird mit einer Steckverbindung auf die Grundplatte gesteckt. Auf der Unterseite der Grundplatte wir die Halterung für den Ultraschallsensor mit vier Schrauben befestigt. Zusätzlich sind zwei kleine Winkel aus Aluminium angebracht, um Schalter für die Mikrocomputer zu befestigen. Die Kabelführung ist ebenfalls mithilfe von Klebesockel auf der Grundplatte gelöst.



\subsubsection{Verbindungswinkel}
\label{sssec:VerbindungswinkelBeschrieb}
Der Verbindungswinkel ist die Verbindung der Grundeinheit und der Antriebseinheit. Dabei wird der Verbindungswinkel mit zwei M5 Schrauben auf an die Grundplatte geschraubt. Der Verbindungswinkel besitzt eine Gleitlagerung aus Messing.

\subsubsection{Kamerahalterung}
\label{sssec:Kamerahalterung}
Die Kamera ist etwa 20 Zentimeter vor der Seilwinde für den Hub angebracht. Dafür wurde ein 11 Zentimeter langes Winkelprofil aus Aluminium an der Grundplatte angeschraubt. An diesem ist ein Aluminiumwinkel befestigt, der die Kamera hält. Dieser ist schwenkbar, damit die Kamera optimal ausgerichtet werden kann. Dabei ist essentiell wichtig, dass die Grundplatte immer Parallel zum Boden liegt, da sonst die Kamera falsche Daten liefern würde und somit die Genauigkeit beim Absetzen stark beeinträchtigen würde. Die \textit{RasPi}-Kamera wird mit 4 Schrauben an dem Winkel befestigt. Zusätzlich ist auf dem Winkelprofil der Ultraschallsensor in x-Richtung angebracht und auf dem Aluminiumwinkel der Endschalter.



\subsection{Last greifen und absetzen}
\label{ssec:Lastgreifenundabsetzen}
%todo Testprotokolle verlinken
Das Greifen der Last erfolgt über einen gelenkig gelagerten Haken.  Eine nicht-Drehbare Querstange aus Aluminium dient als Halterung für den Haken. Ebenfalls an dieser Querstange ist ein Schwenkbegrenzer mit einer Feder befestigt. Die Feder hält den Haken in einer Schieflage von etwa 45\SI{}{\degree}. Weil die Startposition der Last vorgegeben ist, wird das Aufheben der Last im voraus programmiert.
Nach dem anheben der Last wird die Feder durch das zusätzliche Gewicht gestaucht und der Winkel des Hakens zur Vertikalen verkleinert sich auf etwa 10\SI{}{\degree}.\\
Beim Absetzen der Last auf der Zielplattform drückt die Feder den Haken automatisch wieder in die Ursprungsposition und klinkt sich damit aus dem Haken der Last aus.
Durch einen gekrümmten Draht an der Unterseite des Greifers wird sichergestellt, dass sich die Last nach dem Anheben mittig zentriert und durch die Fahrt nicht verschoben wird.
Der Haken ist aus ABS gefertigt und wird 3D gedruckt.

\subsection{Last anheben und senken}
\label{ssec:Lastanhebenundsenken}
Die Last wird mithilfe einer Seilwinde angehoben, welche über einen Schrittmotor angetrieben wird. Dieser ist auf der Befestigungsplatte positioniert. Auf der Motorachse ist eine Seiltrommel aus Aluminium befestigt, die einen Stofffaden aufwickelt. Das Ende des Fadens ist direkt am Greifer mit einer Schlaufe an einer Schraube befestigt.
Da ein loses Seil stark in Schwingung geraten kann, wird zusätzlich eine Führung mit zwei Stäben aus Kohlenstofffasern verwendet. Auf der Befestigungsplatte sind dafür Führungsrohre angebracht. Um den Widerstand zwischen der Führung und dem Stab so klein wie möglich zu halten, wird das Führungsrohr geschmiert. Aufgrund dieser Führungsstäbe können Schwingungen und Verdrehungen vermieden werden.
Dank diesem Aufbau kann der Greifer genauer geführt werden und somit der Transport der Last optimiert ablaufen.



\subsection{Regelung und Steuerung}
\label{ssec:RegelungundSteuerung}
Als Kern der elektronischen Steuerung wird das \textit{ArduinoDue} Board verwendet, welches mit dem \textit{Atmel ATSAM3X8E} Mikroprozessor bestückt ist. Programmiert wurde mit C++.

Über das Board werden zwei Schrittmotoren-Treiber sowie ein LCD-Display angesteuert. Zur Rückmeldung werden zwei Ultraschallsensoren und ein \textit{Raspberry Pi 3 Model B} (RasPi) verwendet.

Weil die Steuerung vom RasPi die Richtung und Distanz zur Zielplattform übermittelt, kann das System als Regelungssystem betrachtet werden. Solange das Ziel nicht erkannt wird, soll das Gerät mit maximaler Geschwindigkeit fahren. Mit der Erkennung des Ziels wird das Gerät immer langsamer und stoppt über dem Zielfeld. Mit diesem System wird ein effektives Anfahren des Ziels erreicht. Zum Starten und Stoppen des Geräts wird ein Schalter und ein Taster verwendet.

\subsubsection{ArduinoDue}
\label{sssec:ArduinoDue}
%todo Beschreibung Arduino DUE

Als Unterstützung zum \textit{RasPi} wird ein \textit{ArduinoDue} eingesetzt. In Vergleich zum kleineren Arduino Uno liegt der Vorteil in der grösseren Anzahl Ein und Ausgänge, womit es sich vor allem für grössere Projekte anbietet.



\begin{itemize}[noitemsep]
	\item Atmel SAM3X8E ARM Cortex-M3 CPU (32-bit ARM controller)
	\item 512 KB Flash Memory
	\item 54 digital in/-output Pins
	\item 12 analog Input Pins
	\item 4 UART-Schnittstellen
	\item 84 MHz clock
	\item 1 USB OTG Kabelverbindung
	\item 2 Digital zu Analogwandler
	\item 2 TWI
	\item 1 SPI
	\item 2 JTAG
	\item 1 Poweranschluss
	\item Stromversorgung 5V/
\end{itemize}\parencite{ArduinoDue2018}

\subsubsection{RasPi}
\label{sssec:RasPi}
%todo Beschreibung Raspi
Das Raspberry Pi ist ein Einplatinencomputer in der Grösse einer Kreditkarte. Das Ein-Chip-System läuft mit einem ARM-Mikroprozessor von Broadcom. Durch seine Bekanntheit, der vielen Module die es dazu gibt, sowie seinem Preis ist das Raspberry Pi Pionier auf dem Bereich der Einplatinencomputer.

\begin{itemize}[noitemsep]
	\item Quad Core 1.2GHz Broadcom BCM2837 64bit CPU
	\item 1GB RAM
	\item BCM43438 WLAN und Bluetooth Low Energy (BLE) auf dem Board
	\item 40-pin erweitertes GPIO
	\item 4 USB2 Ports
	\item 4 Pol Stereo Ausgang und Composit Video Port
	\item Full-Size HDMI
	\item CSI Kamera Port um eine Raspberry Pi Camera anzuschliessen
	\item DSI Bildschirmausgang um einen Raspberry Pi Touchscreen anzuschliessen
	\item MicroSD Anschluss um das OS und Daten zu speichern
	\item Die geschaltete Micro USB Stromzufuhr wurde auf bis zu 2.5A erhöht
\end{itemize}\parencite{RaspberryPiFoundation2017}

\vspace{1em}

\subsubsection{Kommunikation zwischen Boards}
\label{sssec:Kommunikation}
Zwischen dem Raspberry Pi und dem Elektronikerboard soll der Informationsaustausch über eine serielle Schnittstelle erfolgen.

Für die Verwendung einer Kommunikation mit einer asynchronen, seriellen UART-Schnittstelle können jeweilige PINs verwendet werden.

Wie in Abb. \ref{fig:RaspberryPins} dargestellt sind dies Pin 8 (Tx) und Pin 10 (Rx).
\begin{figure}[h!]
	\centering
	\includegraphics[keepaspectratio, width=0.8\textwidth]{UART_Verbindung_LLS_Steckplatine}
	\caption{Verbindung der Boards}
	\label{fig:RaspberryPins}
\end{figure}

Die RX und TX Pins des Raspberry Pi werden mit den jeweils komplementären Pins des Arduino (dh. TX und RX) zusammengesteckt.

UART muss auf dem Raspberry Pi manuell aktiviert werden.  Zusätzlich muss das integrierte Bluetooth-Modul deaktiviert werden, sodass die genannten Ports nicht \textquotedblleft miniUART\textquotedblright\ verwenden.

Für die serielle Übertragung müssen Baudrate, Port, Databits, Parity Check, Stopbits und flow control definiert werden. Die Programm\-ierung dieser wird auf dem RaspberryPi mit Hilfe des Python Moduls \textquotedblleft pySerial\textquotedblright\ realisiert.

Auf dem LCD-Display werden die Koordinaten in x-, und in z-Richtung dargestellt. Zwei Ultraschallsensoren, je einer in x-, und z-Richtung, messen die jeweiligen Distanzen. Bei der z-Koordinate muss jedoch noch ein Betrag abgezogen werden, da der Ultraschallsensor an der Grundplatte befestigt ist und die angezeigten Koordinaten die der transportierten Last sein müssen. Wie viel abgezogen werden muss, wird über den Hubschrittmotor berechnet. Damit bei der Fahrt über die Hindernisse nicht ein falscher Wert angezeigt wird, werden plötzliche grosse Änderungen nicht berücksichtigt und der vorherige Wert wird angezeigt. Der LCD-Display ist in Fahrtrichtung rechts mit Schrauben an einer Halterung aus dem 3D-Drucker befestigt. Die Halterung wiederum ist auf die Grundplatte gesteckt. Die Masse werden ungefähr einmal pro Sekunde aktualisiert und diese werden in Zentimeter auf eine Kommastelle genau angezeigt. Die Positionsanzeige wird über das Arduino DUE angesprochen.

\subsection{Zielerkennung}
\label{ssec:Zielerkennung}
\subsubsection{Systemübersicht}
\label{sssec:Systemuebersicht}

\begin{figure}[h!]
	\includegraphics[keepaspectratio,width=\textwidth]{TargetRecOS}
	\caption{Systemübersicht}
	\label{fig:Systemuebersicht}
\end{figure}

Auf dem Raspberry Pi wird als Grundinstallation das Betriebssystem Raspian installiert. Darauffolgend werden Module für Python (Version 2) und OpenCV (Version 3) für die Bilderkennung hinzugefügt.
Um auf das Raspberry Pi zugreifen zu können, wird einerseits PuTTy benutzt um eine SSH Verbindung aufzubauen, andererseits benötigen wir den X Server für das Streaming des Kamerabildes.
Um den aktuellen Stand (das Abbild, Image) des RaspberryPi zu speichern und gegebenenfalls wiederherstellen zu können, wird die SD Karte in ein Studentennotebook gelegt und mithilfe Win32 Disk Imager kopiert.

Auf dem jeweiligen Notebook wird PyCharm als IDE benutzt und als Programmiersprache Python verwendet.

Das System muss den Ablauf in Abbildung \ref{fig:AblaufZielerkennung} umsetzen können.

\begin{figure}[h!]
	\centering
	\includegraphics[keepaspectratio,height=0.4\textheight]{Ablaufdiagramm_ModusOperandi}
	\caption{Ablaufdiagramm der PiSensor Logik}
	\label{fig:AblaufZielerkennung}
\end{figure}

Der Schritt \textquotedblleft Zielplatte erkennen\textquotedblright\ wird im Kapitel \ref{sssec:Bilderkennung} ausführlich beschrieben.

Jene Schritte mit der Bezeichnung \textquotedblleft Kommunikation ET\textquotedblright\ werden im Kapitel \ref{sssec:Kommunikation} detailliert beschrieben.

\subsubsection{Bilderkennung}
\label{sssec:Bilderkennung}

\begin{figure}[h!]
	\centering
	\includegraphics[keepaspectratio,width=\textwidth]{BilderkennungAblauf}
	\caption{Aufnahmen der einzelnen Schritte der Zielerkennung}
	\label{fig:AufnahmeZielerkennung}
\end{figure}

Die Erkennung (in Abbildung \ref{fig:AufnahmeZielerkennung} dargestellt) läuft wie folgt ab:

\begin{itemize}[noitemsep]
	\item[-] Videoframe aufnehmen
	\item[-] Frame in Graustufen umwandeln
	\item[-] Helligkeit durch CLAHE\footnotemark normalisieren
	\item[-] Binären Schwellenwertfilter anwenden
	\item[-] Konturen ausfindig machen
	\item[-] Polygone ausfindig machen
	\item[-] Nur Rechtecke weiter beachten
	\item[-] bestimmte Anzahl Rechtecke mit dem gleichen Mittelpunkt ausfindig machen
	\item[-] Zielplatte erkannt
\end{itemize}
\footnotetext{contrast limited adaptive histogram equalization}

Dieser Ablauf wird in Abbildung \ref{fig:ZielerkennungAblauf} noch genauer dargestellt.

\begin{figure}[h!]
	\centering
	\includegraphics[keepaspectratio,height=0.4\textheight]{ZielplatteErkennen}
	\caption{Ablaufdiagramm Zielerkennung}
	\label{fig:ZielerkennungAblauf}
\end{figure}

\subsubsection{Systemarchitektur}
\label{sssec:SysArch}

Die Architektur des RaspberryPi-Sensors wurde in drei Hauptmodule unterteilt (siehe Abb. \ref{fig:SystemArchitektur}). Zum Einen ist dies die Kamera mit der dazugehörenden Logik, wie auch die serielle Schnittstelle über die die Kommunikation realisiert wird, zum Anderen ist dies der Controller, welcher als Supervisor die ganzen Prozesse überwacht und koordiniert.

\begin{figure}[h!]
	\centering
	\includegraphics[keepaspectratio,width=0.5\textwidth]{SystemArchitektur}
	\caption{Modulübersicht der PiSensor Architektur}
	\label{fig:PiSensorSystemArchitektur}
\end{figure}

Die Logik der Erkennung wurde in zwei Teile separiert, bildlich dargestellt in der Abbildung \ref{fig:KameraSensor}. Zum Einen das Erfassen der Daten des Sensors und der Erkennung der Zielplattform, und zum Anderen das Berechnen der Distanz zwischen Zielplattform und der Kamera. Die Schnittstelle ausserhalb des Moduls bildet dabei der 'Sensor Controller', dessen Realisierung wird über eine Factory-Klasse bereitgestellt. Die eigentliche Logik findet in Subklassen der 'TargetRecognition' und 'DistanceCalculation' statt.

Im Moment wird die Erkennung der Zielplattform (wie in Kapitel \ref{sssec:Bilderkennung} dargestellt) über einen Kontur\-erkennungs\-algorithmus der Bibliothek 'OpenCV' realisiert. Ist die Zielplattform erkannt, so wird über den Controller das Signal zum Berechnen der Distanz gegeben. Die Methodik zur Distanzberechnung wird in Kapitel \ref{ssec:DistModelCreation} genauer erläutert.

Ist die Entfernung zur Zielplattform ermittelt, so soll der PiController genaue Fahrtanweisungen zur Motor- und Sensorsteuerung übergeben. Dies wird über das 'communicator'-Modul realisiert.

\begin{figure}[h!]
	\centering
	\includegraphics[keepaspectratio,width=0.5\textwidth]{pi-sensor_cs}
	\caption{Gesamtübersicht der Erkennungslogik}
	\label{fig:KameraSensor}
\end{figure}

\begin{landscape}

	\begin{figure}[h!]
		\centering
		\includegraphics[keepaspectratio,width=\linewidth]{pi-sensor}
		\caption{Detailansicht des Raspberry Pi Sensormoduls}
		\label{fig:PiSensor}
	\end{figure}

\end{landscape}

\subsubsection{RasPi-Kamera}
%todo Beschreibung Kamera
Für die Erkennung der Zielplattform wird die \textit{RasPi}-Kamera v2.1 verwendet. Angebracht ist die Kamera zuvorderst am Gerät mit vier Schrauben an einem Winkel. Damit die Zielplattform genügend früh erkannt werden kann und die Last nicht das Bild stört, ist die Kamera mit einem Winkel \SI{20}{\degree} zur Vertikalen nach vorne gerichtet. Das Bild wird auf dem \textit{RasPi} ausgewertet und erkennt damit die Zielplattform.



Wie in Abb. \ref{fig:SchemaKommunikation} dargestellt sind dies Pin 18 (Tx) und Pin 19 (Rx) auf dem Arduino und GPIO14 und GPIO15 auf dem Raspberry Pi. Zusätzlich werden noch die jeweiligen GND (Ground) verbunden.
\begin{figure}[h!]
	\centering
	\includegraphics[keepaspectratio, width=0.8\textwidth]{SchemaKommunikationArduinoRaspbiScreenshot}
	\caption{Verbindung der beiden Boards}

	\label{fig:SchemaKommunikation}
\end{figure}



\subsection{Positionsanzeige}
\label{ssec:Positionsanzeige}

Auf dem LCD-Display werden die Koordinaten in x-, sowie z-Richtung dargestellt. Zwei Ultraschallsensoren, je einer in x-, und z-Richtung, messen die jeweiligen Distanzen. Bei der z-Koordinate muss jedoch noch einen Betrag abgezogen werden, da der Ultraschallsensor an der Grundplatte befestigt ist und die angezeigten Koordinaten die der transportierten Last sein müssen. Wie viel abgezogen werden muss, wird über den Hubschrittmotor berechnet. Damit bei der Fahrt über die Hindernisse nicht ein falscher Wert angezeigt wird, werden plötzliche grosse Änderungen nicht berücksichtigt und der vorherige Wert wird angezeigt. Der LCD-Display ist in Fahrtrichtung rechts mit Schrauben an einer Halterung aus dem 3D-Drucker befestigt. Die Halterung wiederum ist auf die Grundplatte gesteckt.

\newpage

\subsection{Ultraschallsensor}
\label{ssec:Ultraschallsensor}
%todo Quelle Datenblatt
Für die Distanzmessung werden in x-Richtung und y-Richtung je ein Ultraschallsensor verwendet. Die Ultraschallsensoren des Typs HC-SR04 weisen eine Messdistanz zwischen zwei Zentimeter und bis zu fünf Meter auf. Die Genauigkeit wird mit drei Millimetern Abweichung angegeben. Dies ist ausreichend für die geplante Anwendung. Ausgelesen und verarbeitet werden die Informationen auf dem \textit{RasPi}.



\subsection{Endschalter}
\label{ssec:Endschalter}
Der Endschalter ist zuständig, dass das Antriebsrad automatisch stoppt wenn der Zielbereich erreicht ist und der Masten berührt wird. Dafür wird der Endschalter mit dem Arduino ausgelesen. Angebracht ist er am vordersten Punkt des Kamerahalters. Verwendet wurde ein Endschalter des Typs 5E4T125 von Burgess.


\section{Konzeptänderungen nach PREN01}
\label{sec:Konzeptaenderungen}
% TODO Basil : Check Akkus? Check Beschleunigungssensor?

\subsection{Antriebseinheit und Grundplatte}

\begin{itemize}
		\item Das angetriebene Laufrad ist neu aus Gummi und nicht nur mit Gummi beschichtet. Die Reibung auf dem Seil wird damit vergrössert und verhindert Rutschen.
		\item Bei dem Gelenk der Aufhängung ist zusätzlich eine Messingbuchse eingebaut damit möglichst wenig Reibung auftritt.
		\item Die Dämpfer um Schwingungen aufzufangen werden nicht benötigt, da die Schwingungen auch ohne Dämpfer klein sind.
		\item Die Grundplatte bearbeitungstechnischen Gründen nicht mehr aus Aluminium gefertigt sondern aus einer gelaserten, 6mm dicken Holzplatte.
		\item Die Positionsanzeige in x-Richtung wird nur noch über einen Ultraschallsensor bestimmt. Der Schrittmotor wird aufgrund der ausreichenden Genauigkeit des Ultraschallsensors nicht mehr benötigt.
		\item Die geplante Durchhangstabelle wird nicht miteinbezogen, da ebenfalls ein Ultraschallsensor verwendet wird. Der Schrittmotor wird aber noch immer benötigt, da der Ultraschallsensor an der Grundplatte angebracht ist aber die Position der Last angezeigt werden muss.
\end{itemize}

\subsection{Hubvorrichtung}

\begin{itemize}
		\item Der Haken ist neu aus Kunststoff aus dem 3D-Drucker gefertigt. Aufgrund einer leicht veränderten Geometrie ist ein 3D-Druck sinnvoller.
		\item Zusätzlich wird der Haken noch von einem Schwenkbegrenzer eingeschränkt. Damit wird sichergestellt, dass der Haken leicht nach hinten gerichtet ist um die Last besser aufheben oder absetzen zu können.
		\item Am Schwenkbegrenzer ist zusätzlich eine kleine Spiralfeder angebracht. Diese bietet den grossen Vorteil, die Last ohne eine Rückwärtsbewegung auf dem Seil abzusetzen.
		\item Eine geplante Querverstrebung zwischen den beiden Führungsrohren aus Kohlenstofffasern wird weggelassen, da bei Testversuchen keine Probleme ohne die Verstrebung aufgetreten sind.
		\item Der Nylonfaden um den Haken zu heben wird ersetzt durch einen Stofffaden. Der Nylonfaden wäre zu steif um diesen kontrolliert auf der Seilwinde aufzurollen.
\end{itemize}

\subsection{Elektronische Komponenten}

\begin{itemize}
		\item Als Kern der elektronischen Steuerung wird neu anstelle des \textit{LoFive} Boards ein \textit{ArduinoDue} Board verwendet. Weil das \textit{LoFive} ein sehr neues Board ist, bestehen im Vergleich zum \textit{ArduinoDue} wenig bis keine Informationen oder Beispiele im Internet dafür. Ein grosser Mehraufwand und Unsicherheit wäre die Folge gewesen.
		\item Aufgrund des Wechsels vom \textit{LoFive} auf das \textit{ArduinoDue} ist auch die Programmiersprache nicht mehr Rust, sondern es wird neu auf C programmiert.
		\item Die Auswertung der Daten der Ultraschallsensoren erfolgt neu über das \textit{RasPi} anstatt dem \textit{Arduino}. Grund dafür ist die bessere Prozessorleistung des \textit{RasPi}.
\end{itemize}


\section{Schnittstellen}
\label{sec:Schnittstellen}

\subsection{Ultraschallsensoren}
\label{ssec:UltrasonicSensorInterface}

Zur Distanzmessung wird der Ultraschallsensor HC-SR04 gebraucht. Dieser wird über 5V gespeist und über zwei GPIO-Pins angesprochen. Einer davon (Trigger) dient dabei zum Auslösen des Pulses, während der Andere (Echo) ein Signal abgibt, falls der Puls wieder empfangen wurde.

Aus der Zeit zwischen Puls und Empfang kann danach die Distanz über die folgende Formel berechnet werden:

\begin{equation*}
	distanz = \frac{\Delta_{Zeit} \cdot 34300}{2}
\end{equation*}

\section{Bedienungsanleitung}
\label{sec:Bedienungsanleitung}
%todo anpassung höhe des Hakens ab Boden
Um die Funktionalität des Geräts zu gewährleisten, müssen einige Punkte in der Vorbereitung beachtet werden.

\begin{itemize}
	\item Es muss sichergestellt sein, dass alle Kabel eingesteckt sind ausser der Stecker der Energieversorgung von den Akkupacks.
	\item Die Laufkatze muss vorsichtig in das Seil eingehängt werden. Wichtig ist, dass sich die Führungsrollen unter und das Antriebsrad über dem Seil befinden.
	\item Das Gerät muss sich an der Position auf dem Seil befinden, damit der Masten gerade berührt wird.
	\item Die vorgegebene Höhe von 5 Zentimeter vom untersten Punkt des Hakens bis zum Boden muss eingestellt werden. Wenn sich der Haken an der korrekten Höhe befindet, muss der Stecker der Akkupacks am ET-Board eingesteckt werden damit der Schrittmotor die hält.
	\item Der Kippschalter am ET-Board muss nun betätigt werden um das Gerät zu starten.
\end{itemize}

\section{Schnittstellen und Abhängigkeiten}
\label{sec:SchnittAbhang}
% TODO Überprüfen

Die Schnittstellen stellen sicher, dass die voneinander abhängigen Funktionen aufeinander abgestimmt funktionieren. Zum einen ist dies die Kommunikation zwischen dem \textit{RasPi} und dem \textit{ArduinoDue}, zum anderen müssen die physikalischen Komponenten und die Steuerung miteinander verbunden sein. Die Schnittstelle zwischen den Mikrocomputer wird mit einer UART Steckverbindung realisiert. Es ist essenziell, dass das Protokoll auf den Mikrocomputern definiert ist, damit die Informationen richtig übermittelt werden. Der Vorteil gegenüber dem USB liegt in der einfachen Ausführung und es wird kein Adapter für das \textit{ArduinoDue} benötigt.

Viele Schnittstellen sind von der Positionsbestimmung abhängig. Um die Position zu bestimmen, werden die Schrittmotoren und zwei Ultraschallsensoren eingesetzt, welche die Daten liefern. Von der Positionsbestimmung abhängig ist das Greifen der Last, das Heben und Senken derselben und das Absetzen. Da ein Haken eingesetzt wird, muss diese Positionsbestimmung genau sein, da sonst der Transport der Last fehlschlagen würde. Die Genauigkeit in z-Richtung muss dabei in einer Toleranz von 0...+20mm liegen. In x-Richtung darf die Genauigkeit ebenfalls etwa 0...+20mm betragen, da sonst der Haken die Last verfehlt. Der LCD-Display gibt zudem die Koordinaten wieder.

Damit die Zielerkennung mit OpenCV funktionieren kann, muss die Plattform möglichst ruhig und waagrecht sein. Dies ist abhängig von der Aufhängung und Antrieb. Die durch die Beschleunigung auftretenden Schwingungen werden mit einer gelenkigen aber gedämpften Aufhängung minimiert und die Beschleunigungskurve des Antriebs ist S-Förmig definiert. Gelenkig muss diese sein, damit die Plattform bei variabler Steigung waagrecht bleibt. Ein zusätzlicher Vorteil ist, dass das Greifen der Last mit dem Haken durch die minimierten Schwingungen begünstigt wird. Unabhängig von den Schwingungen muss auch auf die Anordnung der Komponenten geachtet werden, damit beim Lasttransport nicht eine Schieflage entsteht. Da sich die Komponenten auf die Balance der Plattform auswirken, muss deren Position im voraus evaluiert werden.

Das Startsignal erfolgt mit einem Kippschalter, welcher am Gerät angebracht ist. Diese Variante ist weniger störungsanfällig als eine Signalerkennung via Bluetooth oder mittels einem akustischen Signal.


\chapter{Tests}
\label{sec:TestsPrototyp}

Im Rahmen des Projektes wird das System bzw. Fahrzeug und deren Komponenten fortlaufend
getestet. Die folgenden Testszenarien und Methoden sollen mögliche Fehler am System oder in
Komponenten frühzeitig entdecken und so die Möglichkeiten bieten, Verbesserungen
vorzunehmen. Weiter kann durch definierte Testfälle sichergestellt werden, dass alle essentiellen
Komponenten getestet werden.

\section{Methodik}

Bevor die Komponenten in das Gesamtsystem integriert wurden, wurden sie einzeln getestet. Danach wurden sie ins Gesamtsystem integriert und erneut getestet. Für sämtliche Tests wurde immer zuerst ein Ziel anhand der Anforderungsliste definiert und dann die Resultate mithilfe eines Protokolls festgehalten.

\section{Testszenarien}

\begin{longtable}{|p{.05\textwidth}|p{.05\textwidth}|p{0.2\textwidth}|p{0.25\textwidth}|p{0.25\textwidth}|}
	\hline
	\textbf{Nr.} & \textbf{Anf.\footnotemark}& \textbf{Test} & \textbf{Was wurde getestet} & \textbf{Fazit} \\
	\hline
	2.01 & 3.7 & Schrittmotoren einschalten, ansteuern und ausschalten & Schrittmotoren mit Startschalter einschalten drehen lassen und ausschalten & Motoren drehen wie erwartet \\
	\hline
	2.02 & 3.7 3.8 & Hub- und Fahrmotor ansteuern & Ansteuerung beider Schrittmotoren& Beide Schrittmotoren konnten parallel betrieben werden \\
	\hline
	2.03 & 4.6 4.7 4.8 & Prototyp Kommunikation & Ein funktioneller Prototyp der in der Lage ist auf GPIO Port 8(Tx) und 10(Rx) seriell kommunizieren kann & Es wurden zwei Skripte erstellt um die generelle Fähigkeit der Kommunikation zu erarbeiten \\
	\hline
	2.04 & 3.9 & Zielplattform Fläche und Distanz & Ziel des Tests soll sein, die Distanz eines cm’s (in Pixel) unter Berücksichtigung der Höhe zu ermitteln & Es konnte eine Liste mit Pixelwerten von 1 cm aus unterschiedlichen Höhen erstellt werden \\
	\hline
	2.05& 3.7 & Start Endschalter & Testen der Funktionalität des ET- und Arduinoboards & Funktionen konnten erfolgreich getestet werden \\
	\hline
	2.06 & 3.7 & Schrittmotor Antrieb & Test des Schrittmotors für den Antrieb. Dabei soll die Funktion des Schrittmotors des Antriebes und das Verhalten des Antriebes / Laufkatze soll getestet werden & Die Laufkatze ist gefahren \\
	\hline
	2.07 & 4.6 4.7 & Ultraschall LCD & Arduino UNO Code zur Ansteuerung des Sensors und LCDs & Der Sensor ist auf die Wettbewerbsdistanz genügend genau\\
	\hline
	2.08 & 4.6 4.7 & Ultraschall DUE LCD & Arduino DUE Code zur Ansteuerung des Sensors in x-Achse(Länge) und z-Achse(Höhe) und des LCD-Displays mit Anzeige von beiden Längen & Der Sensor konnte auch auf dem Arduino DUE betrieben werden, jedoch blockiert er sich teilweise selber. Die Genauigkeit der Messung konnte mithilfe eines Röhrchens das über den Sender gestülpt wurde verbessert werden.\\
	\hline
	2.09 & 3.7 3.8 4.5 & Starten und Last anheben & Der ganze Aufbau soll vom Start eine bestimmte Länge vorwärts fahren bis zur Last. Dort angekommen soll der Aufbau die Last greifen und hochziehen. Dabei fährt der Aufbau weiter vorwärts. Die Last soll sicher gegriffen werden und vor dem erstmöglichen Hindernis hochgezogen werden & Das vorwärtsfahren, das greifen und heben hat funktioniert. Jedoch wurde die Last zu wenig schnell gehoben, weshalb das Hindernis berührt wurde.\\
	\hline
	2.10 & 3.8 4.8 & Last absetzen & Absetzen der Last auf einen Vordefinierten Punkt auf den Parkour & Die Last konnte abgesetzt werden, jedoch musste das Anheben der Last neu kalibriert werden, da sich die Hakengeometrie verändert hatte\\
	\hline

\end{longtable}
\footnotetext{Getestete Anforderungen aus Anforderungsliste}

Die Detaillierten Versuchsprotokolle sind im Anhang zu finden.

\section{Test Komplett}

\section{Fazit}

Die durchgeführten Tests (siehe Anhang) zeigen auf, dass alle wesentlichen
Komponenten soweit funktionieren und nur in folgenden Fällen zu Komplikationen führen kann:\\

\begin{itemize}
	\item Durch den Parallelbetrieb der Motoren Sensoren und des Displays kam es zu einer Stockenden Fahrbewegung. Dabei stellte sich heraus, dass das Arduino DUE Programme nicht parallel abarbeiten kann. Deshalb wurden die Ultraschallsensoren auf das Raspberry PI ausgelagert. zusätzlich musste die Ansteuerung der Motoren mithilfe einer anderen Funktion gelöst werden.
	\item Der Ultraschallsensor hat sich teilweise selber blockiert, da er nach dem senden des Tastimpulses auf eine Antwort wartete, diese jedoch nie kam. Deshalb musste zusätzlich eine weitere Funktion implementiert werden, sodass wenn der Ultraschallsensor nach 2 Sekunden keine Antwort erhielt eine weitere Messung durchgeführt wurde.
\end{itemize}

\chapter{Essentielle Berechnungen und Resultate}
\label{sec:EssBerechnung}
\section{Auslegung Hubmotor}
\begin{table}[h!]
	\centering
	\begin{tabular}{|c|c|c|c|}
		\hline
		\textbf{Beschreibung}& \textbf{Symbol} & \textbf{Grösse} & \textbf{Einheit} \\
		\hline
		Gewicht Last& m$_{\text{L}}$ & 120 & [g] \\
		\hline
		Gewicht Vorrichtung& m$_{\text{V}}$ & 130 & [g] \\
		\hline
		Hubgeschwindigkeit& v & 0.1 & [$m/s^2$] \\
		\hline
		Radius Seiltrommel & r & 0.015 & [m]\\
		\hline
		Erdbeschleunigung & g & 9.81 & [$m/s^2$]\\
		\hline
	\end{tabular}
	\caption{Geschätzte Einheiten und Konstanten}
\end{table}
\noindent
Winkelgeschwindigkeit\\
$\omega=v/r=6.67s^{-1}$	\\
Moment\\
$\underline{\underline{M}}=(m_L+m_V)*g*r=\underline{\underline{0.0368m/s^2}}$\\
Leistung	\\
$\underline{\underline{P}}=M*\omega=\underline{\underline{0.245W}}$

\section{Grobauslegung Motorenleistung Antrieb}
\label{ssec:GrobMotor}
\begin{table}[h!]
	\begin{tabular}{|p{0.3\textwidth}|p{0.15\textwidth}|p{0.15\textwidth}|p{0.15\textwidth}|}
		\hline
		\textbf{Beschreibung} & \textbf{Symbol} & \textbf{Grösse}& \textbf{Einheit}  \\
		\hline
		Seilwinkel & $\alpha$ & 20 & [\SI{}{\degree}] \\
		\hline
		Seilwinkel & $\beta$ & 0.35 & [rad] \\
		\hline
		Masse & m & 3.5 & [kg] \\
		\hline
		Erdbeschleunigung & g & 9.81 & [$m/s^2$] \\
		\hline
		Radradius & r & 0.03 & [m] \\
		\hline
		Raddurchmesser & d & 0.015 & [m] \\
		\hline
		Rollwiderstandskoeffizient & $\mu$ & 0.35 & [rad] \\
		\hline
	\end{tabular}
	\caption{Grobauslegung Motorenleistung}
	\label{tbl:Motorenleistung}
\end{table}

\noindent
Gewichtskraft\\
$F_{g}=m*g=34.335 N$	\\
Normalkraft	\\
$F_{N}=m*g*cos(\alpha)=32.264 N$	\\
Hangabtriebskraft	\\
$F_{H}=m*g*sin(\alpha)=11.743 N$	\\
Rollwiderstand	\\
$F_{R}=F_{N}*\mu=0.645 N$	\\
Nennmoment	\\
$\underline{\underline{M}}=(F_{R}+F_{H})*r=\underline{\underline{0.186 Nm}}$	\\
Nennleistung	\\
$\underline{\underline{P}}=\Delta E/\Delta t=m*g*\Delta h/\Delta t=(1.5kg*9.81m/s^2*0.5m)/(15s)=\underline{\underline{0.49W}}$\\

\section{Reibwerte (Haftreibung)}
\label{ssec:ReibWer}

\begin{table}[h!]
	\centering
	\begin{tabular}{|p{0.3\textwidth}|p{0.15\textwidth}|}
		\hline
		\textbf{Materialpaarung} & \textbf{$\mu$ trocken}\\
		\hline
		Stahl auf Stahl & 0,3 ... 0,8 \\
		\hline
		Holz auf Stahl & 0,5 ... 0,6\\
		\hline
		Gummi auf Stahl & $>$0,5 \\
		\hline
		Kunststoff auf Stahl & 0,25 ... 0.4\\
		\hline
	\end{tabular}
	\caption{Reibwerte möglicher Materialkombinationen \parencite{Wittel2015}}
	\label{tab:Reibwerte}
\end{table}

\noindent
$\mu=\frac{sin(\alpha)}{cos(\alpha)}$\\

\noindent
daraus folgt\\
$\mu=tan(\alpha)$\\
Wir erwarten einen Winkel zwischen $\alpha_{min}=8$ und $\alpha_{max}=30$ $ \rightarrow \mu = 0,14 ... 0,58$. Daraus folgt, dass wir als Material für die Rollen Gummi verwenden müssen.

\section{Auswertung des Distanzversuchs und Modellermittlung}
\label{ssec:DistModelCreation}
% TODO Write this subsection in full glorious detail

Da wir unsere Kamera mit einem Anstellwinkel von \SI{70}{\degree} auf die Ebene schauen lassen, verzerren wir das Bild (und vor allem die Distanzen) perspektivisch (skizziert in Bild \ref{fig:Perspektive}). Um diese korrekt berechnen zu können wurde aus verschiedenen Höhen, die Länge in cm und in Pixel gemessen und durch die gesammelten Daten ein Modell gelegt. Dies wurde im Programm R Studio bewerkstelligt. Leider konnten die Daten nicht durch ein lineares Modell approximiert werden, und es musste ein Quadratisches verwendet werden, dies ist in Abbildung \ref{fig:Modelle} zu sehen.

Jedoch konnte auch keine einzelne Funktion für alle Höhen verwendet werden. Es müssen dadurch auch die Koeffizienten durch eine Funktion approximiert werden. Wir erhalten schlussendlich das Modell abhängig von Höhe h in cm und Distanz in Anzahl Pixel:

\begin{align*}
	\begin{split}
		Distanz_{cm} &= a\cdot x^2 + b\cdot x + c
		\\
		x &= Distanz_{Pixel}
		\\
		a &= 0.000115
		\\
		b &= 0.002585856 \cdot h - 0.064546103
		\\
		c &= -\frac{2}{75} \cdot h + \frac{3}{5}
	\end{split}
\end{align*}

\begin{figure}[h]
	\centering
	\includegraphics[keepaspectratio, width=0.8\textwidth]{Perspektive}
	\caption{Perspektivische Verzerrung der Abbildung von jeweils \SI{10}{\degree}-Bögen}
	\label{fig:Perspektive}
\end{figure}

\begin{figure}[h]
	\centering
	\begin{subfigure}[b]{0.3\textwidth}
		\includegraphics[width=\textwidth]{hoehe52_5}
	\end{subfigure}
	\quad
	\begin{subfigure}[b]{0.3\textwidth}
		\includegraphics[width=\textwidth]{hoehe57_5}
	\end{subfigure}
	\quad
	\begin{subfigure}[b]{0.3\textwidth}
		\includegraphics[width=\textwidth]{hoehe62_5}
	\end{subfigure}
	\hfill
	\begin{subfigure}[b]{0.3\textwidth}
		\includegraphics[width=\textwidth]{hoehe67_5}
	\end{subfigure}
	\quad
	\begin{subfigure}[b]{0.3\textwidth}
		\includegraphics[width=\textwidth]{hoehe72_5}
	\end{subfigure}
	\quad
	\begin{subfigure}[b]{0.3\textwidth}
		\includegraphics[width=\textwidth]{hoehe77_5}
	\end{subfigure}
	\hfill
	\begin{subfigure}[b]{0.3\textwidth}
		\includegraphics[width=\textwidth]{hoehe82_5}
	\end{subfigure}
	\caption{Modelle der unterschiedlichen Messhöhen, jeweils [cm] auf der y-Achse und [Pixel] auf der z-Achse}
	\label{fig:Modelle}
\end{figure}

\chapter{Projektorganisation}
\subsection{Einleitung}
%TODO Schreiben

\section{Grober Rahmenplan}
\label{sec:GrobRahmenplan}
An dieser Stelle ist der Projektplan grob umrissen, den Detaillierten finden Sie im Anhang.

\begin{figure}[h!]
	\includegraphics[width=\linewidth,keepaspectratio]{Rahmenplan}
	\caption{Grober Projektplan PREN02}
	\label{fig:GrobProjektReal}
\end{figure}

\section{Effekte der Massnahmen zur Risikoreduktion}

Im Kapitel \ref{ch:RisikoMgmt} wird der Risikokatalog und Massnahmen genauer beleuchtet. Hier nur beispielhaft der Effekt für zwei Risiken und deren Entwicklung aufgrund getroffener Massnahmen.

\begin{figure}[h!]
	\includegraphics[width=\linewidth,keepaspectratio]{RisikomatrixSquashed}
	\caption{Effekte der Massnahmen beispielhaft}
	\label{fig:RiskMatSquashed}
\end{figure}


\chapter{Schlussdiskussion}
\label{ch:SchlussDisku}
Die Zusammenarbeit im Team war trotz einigen grösseren Schwierigkeiten bis zum Schluss sehr gut. Das Team war immer in der Lage sich gegenseitig zu unterstützen und Probleme zu beheben. Dies war unbedingt nötig, da Eve Meier (Informatik) nach dem Pren1 das Team verlassen hat und sich David Craven (Elektrotechnik) nach kurzer Zeit im Pren2 abgemeldet hat. Dadurch wurde das Team sehr unausgeglichen mit vier Maschinentechnik-Studenten und nur je einem Informatik-, und Elektrotechnikstudenten. Die Studenten der Maschinentechnik versuchten zu helfen, wo es möglich war. Zum einen durch mithilfe beim Programmieren oder durch Anpassungen am Gerät um den Programmieraufwand zu reduzieren.\\
Nachfolgend werden nach Fachbereich aufgetrennt, die jeweiligen Erkenntnisse aus dem PRENs-Modul beschrieben.

\section{Maschinentechnik}

\begin{itemize}
	\item Kabelführung\\
	Die Kabelführung wurde sehr lange vernachlässigt und wurde nicht als Problem betrachtet. Vor allem die Anordnung der Komponenten würde einiges einfacher fallen, wenn die Kabelführung schon definiert gewesen wäre. Zudem mussten Kabel verlängert werden aufgrund von Änderungen an der Positionierung einzelner Komponenten.
	\item Ästhetik\\
	Bei der Planung und Konstruktion des Gerätes wurde nicht auf das Aussehen geachtet. Die Denkweise war, dass die Ästhetik gegen Ende des Projektes ein Thema werden würde. Es hat sich aber herausgestellt, dass im Nachhinein eine gute Lösung schwer zu finden ist. Viel einfach wäre es gewesen, wenn schon seit beginn des Projektes Gedanken zum Aussehen gemacht worden wären.
	\item Fertigung\\
	Zu beginn des Projektes wurde ohne grosse Überlegungen Aluminium für viele Bauteile verwendet. Es wäre sinnvoller gewesen, sich im Voraus über die Werkstoffwahl Gedanken zu machen, um Mehraufwand durch mehrmaliges Fertigen zu vermeiden. Zusätzlich wurde die zur Verfügung gestellte Zeit für den 3D-Drucker, den Laser und die Mechanik-Werkstatt zu wenig genutzt.
\end{itemize}

\section{Elektrotechnik}
\begin{itemize}
	\item Wechsel des Boards\\
	Es zeichnete sich schon früh ab, dass die Arbeit mit dem \textit{LoFive}-Board nicht wie gewollt funktioniert. Der später erfolgte Wechsel auf das \textit{Arduino} hätte schon viel früher passieren sollen. Die vorgegeben Meilensteine bei der Programmierung des \textit{LoFive}-Boards wurden selten eingehalten und verlängert. Wegen dem Wechsel des Boards, konnte auch das bisher Programmierte nicht mehr verwendet werden. Dies hat im Endeffekt dazu geführt, dass der komplette Code in kurzer Zeit neu geschrieben werden musste.
	\item Schrittmotoren\\
	Der benötigte Prozessoraufwand für die Ansteuerung der Schrittmotoren wurde unterschätzt und deshalb nicht als mögliches Problem betrachtet. Die Konsequenz daraus ist, dass beim Fahren ein Ruckeln auftritt. Eine mögliche Lösung wäre ein zweiter Mikroprozessor, welcher nur für die Ansteuerung der Schrittmotoren zuständig ist. Aus Zeitgründen musste aber auf diese Verbesserung verzichtet werden.
\end{itemize}

\section{Informatik}
\begin{itemize}
	\item Prototypen\\
	Das IT-Team hat schon früh damit begonnen Prototypen, für die verschiedenen Teilfunktionen die sie betrafen, zu entwickeln. Dies hatte zur Folge, dass schon früh grundlegende Fehler erkannt wurden und das IT-Team besser für die Eventualitäten, die im Projekt auftraten (Veränderte Teaminterne Anforderungen, mehr Ressourcen für Organisatorisches) bereit war.

	\item Boardwahl\\
	Im Nachhinein wäre es besser gewesen, ein gemeinsames Board für die Entwicklung zu verwenden. Dadurch wäre eine Schnittstelle (Kommunikation) weggefallen und es wären mehr Ressourcen (/Entwickler) zur Verfügung gestanden, die man frei hätte verteilen können. So wie es gelöst wurde, (zwei verschiedene Boards) entstand beim Helfen ein Verlust beim Beschaffen und Einarbeiten, welchen man hätte vermeiden können.

\end{itemize}

\section{Arbeitsprozess}
	Die Zusammenarbeit unter den Disziplinen hat aufgrund der Anwesenheit an beiden Schultagen gut funktioniert. Lösungen wurden, soweit als möglich, gemeinsam gesucht und auch gefunden. Einige Probleme hätten aber verhindert werden können, wenn die gesetzten Meilensteine und die Konsequenzen, falls nicht erreicht, eingehalten worden wären. Ein strikteres Vorgehen wäre rückblickend wünschenswert gewesen. Zudem wurde hauptsächlich am Anfang des Projektes viel Zeit mit diskutieren in der Gruppe über Details verbraucht, welche zu diesem Zeitpunkt noch nicht relevant waren.

\section{Eingetretene Risiken}
\label{ch:EingetreteneRisikos}

Die Risiken beziehen sich auf die Tabelle im Kapitel \ref{sec:Risikokatalog}.

\subsection{Ausfall Teammitglied}
Das Risiko Nr. 10 ist eingetroffen. Nach der vierten Semesterwoche des PREN2 fiel ein Teammitglied aufgrund interner Differenzen aus. Da es sich bei diesem Teammitglied um einen Elektrotechnikstudenten handelt, ist dieser Ausfall für unser Team besonders schmerzhaft, da wir nun nur noch einen Elektrotechnik-Student haben. Dieses eingetroffene Risiko bedeutete nun, dass der verbleibende Elektrotechnik-Student für die ganze Arbeit in seinem Fachbereich zuständig war, was sehr viel zusätzliches Engagement bedeutete.

\subsection{Änderung Mikrokontrollerplattform}
\label{ch:AenderungMikrokontroller}
Das Konzept vom PREN1 beinhaltet als Mikrokontrollerplattform das "LoFive" Board. Wie sich jedoch herausstellte, war es für uns unmöglich diesen Mikrokontroller für unsere Aufgaben zu programmieren. Aus diesem Grund wurde in Semesterwoche drei der Entscheid gefällt, dieses durch einen anderen Mikrokontroller, also durch das \textquotedblleft Arduino Due\textquotedblright\ zu ersetzen. Das eingetretene Risiko beinhaltet viele nicht brauchbare bzw. nur halbwegs brauchbare Arbeitsstunden des Programmieren für das "LoFive".

\section{Fazit}
Trotz erschwerten Bedingungen, wie den Austritt von gleich zwei Teammitglieder oder den Wechsel des ET-Boards im Pren 2, konnte das Projekt insgesamt erfolgreich abgeschlossen werden. Ein Grund dafür war das bestreben, eine möglichst einfache Lösung zu finden. Da dies gelang, konnte eine Kombination gefunden werden, die im Vergleich zu anderen Möglichkeiten wenig Aufwand bereiteten und dennoch zuverlässig funktioniert.

Der Lerneffekt für das Team in Bezug das auf Projektmanagement und vor allem das interdisziplinäre Arbeiten war gross und die gesammelten Erfahrungen sicherlich nützlich für zukünftige Projekte. Das Arbeiten im Team hat allen Mitgliedern gut gefallen und alle konnten ihr Wissen und ihre Fähigkeiten einbringen.\\

Das Budget konnte problemlos eingehalten werden. Aufgrund von noch verfügbarem Budget hätte auch im Falles eines Defektes Ersatzteile besorgt werden können.\\
Die vorgegebenen Meilensteine konnten immer eingehalten werden. Die vom Team selbst gesetzten Meilensteine konnten teilweise nicht am vorgegebenen Zeitpunkt erreicht werden.


\listoffigures

\listoftables

\printbibliography

\appendix

\chapter{Aufgabenstellung im Original}
\label{app:ch:AufgabenOriginal}
Die Aufgabenstellung im Original folgt auf den nächsten Seiten.

\includepdf[pages=1-8]{AufgabenstellungPREN1H17.pdf}

\includepdf[pages=1-4]{AufgabenstellungPREN2F18V2.pdf}

\chapter{Anforderungen aus PREN 01}
\label{app:ch:Anforderungen}
\section{Projektanforderungen}
\label{app:sec:ProjektAnf}
\begin{tabular}{|p{.05\textwidth}|p{0.07\textwidth}|p{0.3\textwidth}|p{0.55\textwidth}|}
	\hline
	\textbf{Nr.} & \textbf{FMW\footnotemark} & \textbf{Bezeichnung} & \textbf{Beschreibung} \\
	\hline
	1.1 & F & Projektabgabe & Juni 2018 \\
	\hline
	1.2 & F & Eigenleistung & Systemkomponenten können zugekauft werden \\
	\hline
	1.3 & F & Interdisziplinarität & Disziplinen / Abteilungen arbeiten zusammen \\
	\hline
	1.4 & W & Lieferantenwahl & Für Sammelbestellungen gem. Kapitel 4.5 der Aufgabenstellung. Wird Material vom Team selbst gekauft, können die Kosten zurückgefordert werden \\
	\hline
	1.5 & M & Budget f. PREN & max. 500.- CHF \\
	\hline
	1.6 & M & Teilbudget PREN1 & max. 200.- CHF \\
	\hline
	1.7 & M & 3D-Drucker Laufzeit & max. 25h \\
	\hline
	1.8 & M & Lasergerät Laufzeit & max. 1h \\
	\hline
	1.9 & M & Stunden ET-Werkstattpersonal & max. 10h \\
	\hline
	1.10 & M & Stunden M-Werkstattpersonal & max. 10h \\
	\hline
	1.11 & F & "Gesponsorte"\ Komponenten & Werden mit einem realistischen Preis in die Kostenrechnung einbezogen \\
	\hline
\end{tabular}
\footnotetext{F: Festanforderung M: Mindestanforderung W: Wunschanforderung }

\section{Plattform}
\label{app:sec:Plattform}
\begin{tabular}{|p{.05\textwidth}|p{0.07\textwidth}|p{0.3\textwidth}|p{0.55\textwidth}|}
	\hline
	\textbf{Nr.} & \textbf{FMW\footnotemark} & \textbf{Bezeichnung} & \textbf{Beschreibung} \\
	\hline
	2.1 & F & Gesamtstrecke & 350 $\pm$ 2cm \\
	\hline
	2.2 & F & Masten Abstand & Abstand zwischen den Masten 350cm $\pm$ 2cm \\
	\hline
	2.3 & F & Masten Masse & 8cm Front, 6cm Tiefe \\
	\hline
	2.4 & F & Drahtseil & Verzinkter Stahl, Durchmesser 3mm \\
	\hline
	2.5 & F & Seilspannung & Via Umlenkrollen durch ein Gewicht mit einer Masse von 15kg \\
	\hline
	2.6 & F & Winkel des Seiles & 8-20\SI{}{\degree}\\
	\hline
	2.7 & F & Grundplatte & Spanplatte roh oder grau gestrichen.\\
	& & & Mit Farbresten / vorstehenden Schrauben und Nahtstellen ist zu rechnen \\
	\hline
	2.8 & F & Startfeld & 50cm $\pm$ 2cm, Quadratisch \\
	\hline
	2.9 & F & Zielplatte & 31.5cm x 31.5cm \\
	\hline
	2.10 & F & Zielplatte Aussehen & 5 konzentrische Bereiche?\\
	& & & Der innerste, helle Bereich ist quadratisch und hat eine Seitenlänge von 6cm $\pm$ 2mm. \\
	& & & Jeder daran anschliessende konzentrische Bereich hat eine Breite von 2.5cm $\pm$ 2mm. \\
	& & & Die Bereiche sind abwechslungsweise hell und dunkel \\
	\hline
	2.11 & F & Zielplatte Position & Der Absetzbereich verläuft unterhalb des Seiles und ist 2cm breit.\\
	& & & Die Zielplatte kann bis zum Startsignal verschoben werden.\\
	& & & Befindet sich aber immer im Absetzbereich (siehe Abbildung 1 der Aufgabenstellung) \\
	\hline
	2.12 & F & Start- und Zielplatte & Matt \\
	\hline
	2.13 & F & Hindernisse & Auf der gesamten Plattform können Hindernisse stehen.\\
	& & & Die Hindernisse können sich in einem Bereich von 10cm nach der Startposition der Last bis 20cm vor dem Mast 2 befinden  \\
	\hline
	2.14 & M & Hindernisse Höhe & Die Hindernisse haben eine maximale Höhe von 20cm. \\
	\hline
\end{tabular}
\footnotetext{F: Festanforderung M: Mindestanforderung W: Wunschanforderung }

\section{Laufkatze}
\label{app:sec:LaufKatze}
\begin{tabular}{|p{.05\textwidth}|p{0.07\textwidth}|p{0.3\textwidth}|p{0.55\textwidth}|}
	\hline
	\textbf{Nr.} & \textbf{FMW\footnotemark} & \textbf{Bezeichnung} & \textbf{Beschreibung} \\
	\hline
	3.1 & F & Steuerung & Autonom \\
	\hline
	3.2 & M & Inbetriebnahme & Darf max. 2min dauern \\
	\hline
	3.3 & M & Startsignal & Darf per Kopfdruck gesendet werden \\
	\hline
	3.4 & M & Geschwindigkeit & Um die Aufgabe zu bewältigen steht der Laufkatze\\
	& & & ein Zeitfensters von 4min zur Verfügung. \\
	\hline
	3.5 & M & Aussendimensionen & Die Laufkatze darf in ihrer Projektion das Startfeld nicht überschreiten. 50cm $\pm$ 2cm x 50cm $\pm$ 2cm \\
	\hline
	3.6 & F & Bauart & Sämtliche Sensorik muss auf dem Gerät selbst montiert sein. \\
	\hline
	3.7 & F & Fahrweise & Das Gerät darf nur das Drahtseil und den zweiten Masten berühren.\\
	& & & Die gesamte Plattform, insbesondere Drahtseil, die Last und die Zielplatte dürfen nicht beschädigt oder sonst irgendwie verändert werden.\\
	& & & Es ist beispielsweise nicht erlaubt, Navigationshilfen anzubringen. \\
	\hline
	3.8 & F & Ladegut & Das Gerät muss ein Ladegut transportieren können. \\
	\hline
	3.9 & F & Zielerkennung & Das Erkennen der Zielplatte muss selbstständig erfolgen\\
	\hline
\end{tabular}
\footnotetext{F: Festanforderung M: Mindestanforderung W: Wunschanforderung }

\section{Ladegut}
\label{app:sec:AnfLadegut}
\begin{tabular}{|p{.05\textwidth}|p{0.07\textwidth}|p{0.3\textwidth}|p{0.55\textwidth}|}
	\hline
	\textbf{Nr.} & \textbf{FMW\footnotemark} & \textbf{Bezeichnung} & \textbf{Beschreibung} \\
	\hline
	4.1 & F & Material & Holz, schwarz gestrichen \\
	\hline
	4.2 & F &  Dimensionen & Seitenlänge 5cm $\pm$ 0.5cm \\
	\hline
	4.3 & M & Gewicht & 50-90g, siehe Sektion \ref{app:sec:RechKlotzMasse} \\ %todo Berechning Klotz?
	\hline
	4.4 & F & Aufnahme & Metallischer, magnetischer Haken oben in der Mitte des Würfels.\\
	& & & Innendurchmesser des Hakens ist 1.3cm $\pm$ 0.1mm\\
	\hline
	4.5 & F & Hindernisse & Das Ladegut darf Hindernisse nicht berühren. \\
	\hline
	4.6 & F & Position & Die Position des Ladegutes muss in Echtzeit angezeigt werden, dass der Schiedsrichter jederzeit die angezeigten Werte gut erkennen kann.\\
	\hline
	4.7 & F & Positionsbestimmung & Die Mitte des Bodens des Ladegutes wird verwendet.\\
	& & & Die Position muss in x- und z-Richtung bestimmt werden (siehe Skizze Anforderungen).\\
	& & & Der Nullpunkt des zu verwendenden Koordinatensystems ist in Abbildung 1 der Aufgabenstellung definiert. \\
	\hline
	4.8 & F & Absetzen & Das Ladegut muss innerhalb des Zielbereiches automatisch abgesetzt werden. \\
	\hline
	4.9 & F & Zielbereich & Der Zielbereich muss automatisch erkennt werden\\
	\hline
\end{tabular}
\footnotetext{F: Festanforderung M: Mindestanforderung W: Wunschanforderung }

\section{Umfeld}
\label{app:sec:Umfeld}
\begin{tabular}{|p{.05\textwidth}|p{0.07\textwidth}|p{0.3\textwidth}|p{0.55\textwidth}|}
	\hline
	\textbf{Nr.} & \textbf{FMW\footnotemark} & \textbf{Bezeichnung} & \textbf{Beschreibung} \\
	\hline
	5.1 & M & Licht & Scheinwerfer welche auf die Wettbewerbsplattformen gerichtet sind, können dazu führen, dass wir mit einer hohen Helligkeit arbeiten müssen \\
	\hline
	5.2 & M & Temperaturen & Bei Lagerung und Betrieb 15-\SI{40}{\degreeCelsius}\\
	\hline
	5.3 & M & Kein Spritzwasser & Für Innenanwendung\\
	\hline
	5.4 & M & Keine Hochspannung & Normale Netzspannung\\
	\hline
	5.5 & M & Raumhöhe & Als Wettbewerbsraum wird eine Normale Raumhöhe (2.5-3m) angenommen.\\
	\hline
\end{tabular}
\footnotetext{F: Festanforderung M: Mindestanforderung W: Wunschanforderung }

\chapter{Versuche}
\label{app:ch:Versuche}

Anhand von Aufgabenstellung, Anforderungen und Risikomanagement ist klar, welche Schnittstellen und/oder Teilkomponenten besonders wichtig sind. Zu den wichtigsten Schnittstellen und/oder Teilkomponenten wurden Versuche durchgeführt, welche entweder eine positive Eigenschaft bestätigen, oder eine negative Eigenschaft hervorheben. Die detaillierten Testprotokolle zu den jeweiligen Versuchen sind im Anhang \ref{app:ch:Versuche}.

\subsection{Aufhängung und Antrieb}
In einem ersten Versuch (Anhang \ref{app:ch:Versuche} Testprotokoll Laufkatze + Antrieb\_V1) wurde festgestellt, dass sich die Konstruktion nicht genügend Symmetrisch verhält. Dadurch hängt die Last nicht direkt unter dem Antrieb, sondern kann sich bis zu 1cm nach vorne und nach hinten versetzen. Ob der Versatz nach vorne oder nach hinten ist, hängt von den einwirkenden Beschleunigungen (positiv oder negativ) ab. So können Probleme bei der Positions- und Zielerkennung entstehen. Es muss ein neuer Prototyp ohne Selbstspannmechanismus konstruiert, hergestellt und getestet werden.

Im zweiten Versuch (Testprotokoll Laufkatze + Antrieb\_V2) wurde dann der Prototyp 2 des Antriebes und der Aufhängung getestet. Das Problem des Versatzes in x-Richtung bzw. Pendeln um die y-Achse war damit behoben. Auch die Einstellung der Spannkraft funktioniert gut, sodass kein Selbstspannmechanismus nötig ist. Die nötige Vorspannkraft, mit welcher die Reibungskraft erhöht werden kann, muss während dem Modul PREN2 noch ermittelt werden.

\subsection{Schrittmotoren}
Zum Fortbewegen sowie zum Anheben der Last werden Schrittmotoren eingesetzt. Damit darauf aufbauend Tests fürs Fahren sowie das Lastanheben möglich sind werden die Motoren mit einem Laboraufbau getestet. Angetrieben werden die Schrittmotoren von einem Microchip, welcher spezifisch für dessen Ansteuerung ausgelegt ist.

Mit dem ersten Aufbau kommt der Motor nicht in Bewegung. Darauf folgend wird die Treiberkomponente mit dem Oszilloskop und der Schrittmotor mit einem anderen Ansteuerungs-Modul getestet. So konnte erkannt werden, dass der Motor einwandfrei funktioniert, die Treiberkomponente jedoch einen Defekt hat. Daher muss für einen weiteren Testlauf eine neue Treiberkomponente angeschafft werden.

\subsection{Durchhang}
\label{ssec:VersDurch}
Durch Anhängen eines Gewichtes in regelmässigen Abständen und über die ganze Strecke verteilt, wurde der jeweilige Durchhang gemessen und danach in einer Excel-Tabelle eingetragen. Durch Erstellen eines Diagramms konnte für die Kurve eine Trendlinie erstellt werden. Aus dieser konnte eine Funktion des Durchhanges in Abhängigkeit der Strecke abgeleitet werden. Mit dieser Funktion (siehe Abbildung \ref{fig:Durchhang_v1}) kann später auch die Position ermittelt werden.
\begin{figure}[h!]
	\centering
	\includegraphics[width=\textwidth,keepaspectratio]{Durchhang_v1}
	\caption{Diagramm zum Durchhang}
	\label{fig:Durchhang_v1}
\end{figure}

Interessant ist, dass die tatsächlichen Werte des Durchhanges schlecht mit den errechneten Werten in Tabelle \ref{tbl:DurchhangRechnung} übereinstimmten. Dies führt daher, dass die Umlenkrollen der Plattform eine grosse Reibung aufweisen und daher die Rechnung mit konstanter Seilspannung eigentlich nicht ganz richtig ist. Die Rollen werden, jedoch von der Modulleitung noch durch reibungsärmere ersetzt. Danach muss die Durchhangsmessung erneut durchgeführt werden. Das detaillierte Protokoll ist im im Anhang \ref{app:ch:Versuche} "Testprotokoll Versuch Durchhang".

\subsection{Last greifen \& absetzen}
\label{ssec:VersLastg}

Für einen Mechanismus um die Last zu Greifen, fiel die erste Wahl ursprünglich auf eine Konstruktion mit Elektromagnet. Diesen Elektromagneten wollten wir selbst bauen. Das Testprotokoll \textquotedblleft Elektromagnet V1\textquotedblright\ (Anhang \ref{app:ch:Versuche}) zeigt, dass die Last zwar angehoben werden konnte, der Elektromagnet jedoch sehr heiss wurde. Dies aufgrund der hohen Leistung von 4W welche durch die Spule fliesst.

Nach dem Versuch \textquotedblleft Elektromagnet V2\textquotedblright\ (Anhang \ref{app:ch:Versuche})wurde klar, dass durch die vorgenommenen Änderungen der Elektromagnet nur noch am Rand eine magnetische Wirkung besitzt.

Um eine Alternative ohne Elektromagnet zu testen, wurde ein Prototyp gebaut um die Teilfunktion \textquotedblleft Last greifen\textquotedblright\ durch einen Haken umzusetzen.

Der Versuch \textquotedblleft Last Greifen mit Haken\textquotedblright\ (Anhang \ref{app:ch:Versuche}) hat gezeigt, dass die gelenkig gelagerte Hakenkonstruktion sich gut eignet um die Last mittig anzuheben. Ausserdem hat der Versuch gezeigt, dass durch Berührung der Last, der Haken nach hinten gekippt wird und die Last dadurch nicht verschoben wird. Das Absetzen der Last wir im Versuchsprotokoll \textquotedblleft Last absetzen mit Haken\textquotedblright\ (Anhang \ref{app:ch:Versuche}) beschrieben. Änderungen an der Geometrie des Greifers sind momentan nicht vorgesehen.

\subsection{Entwicklungsumgebung Informatik}
\label{ssec:VersEntI}
Ziel war es sicherzustellen, dass eine Entwicklungsumgebung zur Bilderkennung mit den bisher vorgesehenen Komponenten möglich ist. Der Versuch war erfolgreich und lieferte alle erwarteten Resultate. Das detaillierte Protokoll findet man im Anhang \ref{app:ch:Versuche} \textquotedblleft Testprotokoll – Entwicklungsumgebung Informatik\textquotedblright.
Zusätzlich wurde eine Installationsanleitung (\textquotedblleft Raspy Install\textquotedblright Anhang \ref{app:ch:Versuche}) erstellt.

\subsection{Test Zielplatten Erkennung Informatik}
\label{ssec:ZielplattenErkennung}
Ziel war es, die Zielplatte mittels der Programmbibliothek OpenCV zu erkennen (Testprotokoll - Erkennung Zielplatte,Anhang \ref{app:ch:Versuche}).
Die Zielplatte konnte grundsätzlich erkannt werden. Es stellt sich jedoch heraus, dass sobald die Lichtverhältnisse stark ändern, eine Erkennung nicht mehr möglich ist. Zudem wird die Zielplatte nicht erkannt, sobald eine der konzentrischen Bereiche / Linien durchtrennt wird. Sobald ein Hindernis, die zu hebende Last oder Teile unseres Fahrzeuges im Weg sind, funktioniert dieser erste Prototyp der Zielplattenerkennung nicht.

Um zu wissen ob und und mit welchen Bildstörungen zu rechnen sein wird, wurde ein weitere Versuch \textquotedblleft Kameraausrichtung\textquotedblright\ (Anhang \ref{app:ch:Versuche}) durchgeführt.

\subsection{Kameraausrichtung}
\label{ssec:VersKamera}
Ziel war es die Kameraausrichtung am Fahrzeug festzulegen und zu wissen ob und mit welchen Bildstörungen zu rechnen sein wird.
Die Kamera wird ausserhalb des Geräts angebracht um eine Überbelichtung des Sensors zu verhindern (bzw. gleichmässige Lichtverhältnisse im ganzen Bild zu erhalten).
Dieser Versuch bestätigte das Risiko, dass die Zielplatte nicht erkannt wird, sobald Störfaktoren wie Teleskoparm und/oder die Last selbst im Bild sind.  Mit der optimalen Ausrichtung der Kamera, können solche Störfaktoren vermindert, jedoch nicht komplett ausgeschlossen werden. Das detaillierte Protokoll ist im Anhang \ref{app:ch:Versuche}  \textquotedblleft Testprotokoll Kameraausrichtung\textquotedblright.

\subsection{Kommunikation ET - IT Board}
\label{ssec:VersKomm}
Ziel war es, die Kommunikation via UART zwischen Elektronik-Board und Informatik-Board sicherzustellen. Sie wird hauptsächlich dazu gebraucht, damit das Informatik-Board die Erkennung der Zielplatte an das steuernde Elektronik-Board weitergeben kann. Der Versuch wurde nicht fertiggestellt und ist Gegenstand von PREN2.

\chapter{Berechnungen}
\label{app:ch:Berechnung}
\section{Teleskop}
\label{app:sec:Teleskjope}
\subsection{Teleskop Variante 1}
\label{app:ssec:TeleskopjeVar1}
\begin{figure}[h]
	\centering
	\includegraphics[keepaspectratio,height=4cm]{Teleskoparm1.JPG}
	\caption{Skizze Verfahrweg}
	\label{fig:Skizze Verfahrweg}
\end{figure}

Durch den Weg der die Last zurücklegen muss (violetter Pfad Abbildung \ref{fig:Skizze Verfahrweg}) ergeben sich zwei Extrempositionen.
(Seildurchhang wurde vernachlässigt)


\begin{itemize}[noitemsep]
	\item Position 2: Teleskoparm gestaucht
	\item Position 3: Teleskoparm maximal ausgefahren
\end{itemize}


\begin{figure}[h]
	\centering
	\includegraphics[keepaspectratio,height=4cm]{Teleskoparm2.JPG}
	\caption{Skizze Teleskop}
	\label{fig:Teleskoparm}
\end{figure}

Dabei wurde festgestellt, dass bei Position 2 wesentlich weniger Platz zwischen last und gerät vorhanden ist als bisher angenommen (Abbildung \ref{fig:Teleskoparm}). Bisher wurde mit einem Teleskop gerechnet, welcher aus 3-4 Segmenten besteht. Jedoch lässt sich dies nicht realisieren, da zu wenig Platz vorhanden ist. Für eine solche Lösung wäre ein Teleskop mit mehr als 8 Segmenten nötig. Dies wurde aufgrund des großen Aufwandes verworfen.

\subsection{Teleskop Variante 2}
\label{app:ssec:Teleskopje2}
Deshalb wurde eine Alternativlösung (Abbildung \ref{fig:Linearaufzug}) erarbeitet, bei der die Führungsstäbe nicht wie beim Teleskop eingefahren wird sondern seitwärts neben dem Antriebsmodul vorbei geschoben wird. Dies vereinfacht die bisherige Konstruktion zwar, jedoch muss der Abstand zwischen den Führungen genügend gross sein, dass die Antriebseinheit dazwischen platz hat.

\begin{figure}[h]
	\centering
	\includegraphics[keepaspectratio,height=4cm]{Teleskoparm3.JPG}
	\caption{Skizze Teleskoparm überarbeitet}
	\label{fig:Linearaufzug}
\end{figure}


\section{Durchhang}
\label{app:sec:Durchhang}
Der Durchhang des Seils ist abhängig vom Gewicht der Laufkatze. Dabei ist es wichtig den Durchhang für das sichere Überfahren der Hindernisse zu kennen. Zudem besteht die Möglichkeit, die Positionsanzeige in vertikaler Richtung als Funktion der zurückgelegten Strecke abzubilden, womit ein zusätzliches Messsystem weggelassen werden kann.

\subsection{Rechnung Durchhang}
\label{ssec:RechDurch}
Die Grundsteigung des Tragseils beträgt 8,2 Grad. Mithilfe von Trigonometrie und der Spannung des Seils konnte der Durchhang in der Mitte des Tragseils näherungsweise ausgerechnet werden.

\vspace{1em}
\noindent
\begin{table}[h!]
	\begin{tabular}{|p{0.15\textwidth}|p{0.2\textwidth}|p{0.55\textwidth}|}
		\hline
		\textbf{Gewicht [kg]} & \textbf{Steigungswinkel [\SI{}{\degree}](ohne Grundsteigung)} &\textbf{Durchhang [cm]}\\
		\hline
		0.5&0,95&2.29\\
		\hline
		1&1,91&5,84\\
		\hline
		1.5&2,87&8,76\\
		\hline
		2&3,82&11,69\\
		\hline
		2.5&4,78&14,63\\
		\hline
		3&5,74&17,59\\
		\hline
		3.5&6,70&20,56\\
		\hline
		4&7,66&23,54\\
		\hline
		4.5&8,63&26,55\\
		\hline
		5&9,59&29,58\\
		\hline
	\end{tabular}
	\caption{Rechnung Durchhang}
	\label{tbl:DurchhangRechnung}
\end{table}


\subsection{Grobversuch Durchhang}
\label{app:ssec:GrobeversDurch}
Es wurde ein einfacher Versuch über den Durchhang des Tragseiles gemacht. Die Dimensionen und Proportionen sind nicht massstäblich, nur Beispielhaft. Dabei wurde ein Seil zwischen zwei Stühlen durch ein Gewicht gespannt. Ein angebrachtes Gewicht wurde dannach an verschiedenen Positionen angehängt und der Durchhang wurde gemessen. Dabei kam raus, dass der maximale Durchhang in der Mitte des Tragseils ist. Der Durchhang im oberen fünftel des Tragseils betrug dabei ungefähr 2/3 des maximalen Durchhangs.
Für eine genaue Funktion des Durchhangs könnten am Musteraufbau Gewichte an verschiedenen Positionen angehängt und der Durchhang von Hand messen. Durch Eingabe in Excel kann somit die Funktion für verschiedene Gewichte ermittelt werden.

\chapter{Meilensteinberichte}
\label{app:ch:MeilensteinBerichte}

\chapter{Projektorganisation und Projektplanung}
\label{ch:ProjektOrga}

\section{Teamübersicht}
\label{sec:Teamuebersicht}
\begin{table}[h!]
	\centering
	\begin{tabular}{|p{0.3\textwidth}|p{0.3\textwidth}|}
		\hline
		\textbf{Name} & \textbf{Studium} \\
		\hline
		Basil Bachmann & Maschinenbau \\
		\hline
		Pascal Baumann & Informatik \\
		\hline
		Victor Guntern & Maschinenbau \\
		\hline
		Markus Kempf & Maschinenbau \\
		\hline
		Jan Odermatt & Elektrotechnik \\
		\hline
		Simon Rohrer & Maschinenbau \\
		\hline
	\end{tabular}
	\caption{Teammitglieder}
	\label{tab:TeamMitglieder}
\end{table}

\newpage

\section{Projektrollen}
\label{sec:ProjektRollen}
\begin{table}[h!]
	\begin{tabular}{|p{0.3\textwidth}|p{0.5\textwidth}|p{0.2\textwidth}|}
		\hline
		\textbf{Rolle} & \textbf{Aufgaben} & \textbf{Teammitglied} \\
		\hline
		Projektleiter & Gesamtübersicht des Projektes halten  & Markus \\
		& Überprüfen ob Vorgaben eingehalten werden & Pascal Stv. \\
		& Teammeetings organisieren & \\
		& Informationsaustausch sicherstellen & \\
		& Kostenverwaltung & \\
		\hline
		Projektplaner & Aktualisieren des Terminplanes & Markus\\
		& Rahmenplanung und Überblick über Einhaltung Meilensteine& \\
		& Pflege Taskboard & \\
		\hline
		Verantwortlicher Dokumentation& Zusammenstellen des Abgabedokuments für Meilensteine & Pascal \\
		& Unterstützung und Pflege LaTeX &  \\
		\hline
		Protokollführer & Protokolle führen & Viktor / Simon \\
		& Alte Protokolle abnehmen lassen & \\
		\hline
		Fachverantwortliche & Projektstand und Feedback & I Pascal \\
		& Ansprechperson bei Fragen & ET Jan\\
		& Aktualisierung Risikomanagement & M Markus\\
		& Koordination Versuche und Recherche & \\
		\hline
	\end{tabular}
	\caption{Rollen in unserem Team}
	\label{tab:Projektrollen}
\end{table}

\section{Tools}
\label{sec:Tools}
\begin{table}[h!]
	\begin{tabular}{|p{0.4\textwidth}|p{0.6\textwidth}|}
		\hline
		\textbf{Aufgabe} & \textbf{Hilfsmittel} \\
		\hline
		Dokumente und Dokumentation & LaTeX / MiKTeX / GitHub \\
		\hline
		Versionierung Dokumentation & git / GitHub\\
		\hline
		Git Clients & GitKraken, SourceTree, GitHub Desktop\\
		\hline
		Quellen & Mendeley \\
		\hline
		Dateiablage für Teamaustausch & Dropbox \\
		\hline
		Dateiablage für Abgabe & Ilias \\
		\hline
		Projekt- und Budgetplan & MS Excel 2016 \\
		\hline
		Kommunikation Team & WhatsApp Gruppe oder über die HSLU Mailadresse\\
		\hline
		Aufgabenverwaltung & SCRUM-angelehntes Board, physisch, Teaminsel\\
		\hline
		Entwicklungsumgebung Arduino & Arduino Studio\\
		\hline
		Entwicklungsumgebung Python & JetBrains Pycharm\\
		\hline
	\end{tabular}
	\caption{Softwaretools}
	\label{tab:SWTools}
\end{table}

\newpage

\section{Wochenplan}
\label{sec:Wochenplan}
\begin{table}[h!]
	\begin{tabular}{|p{0.2\textwidth}|p{0.8\textwidth}|}
		\hline
		\textbf{Tag} & \textbf{Beschreibung} \\
		\hline
		DO 08:30 & Alle Mitglieder sind in der Teaminsel \\
		\hline
		DO 08:30-09:00 & Besprechung Team-intern, Vorbereitung Meeting mit Dozent \\
		\hline
		DO 09:00-09:30& Besprechung mit Dozent \\
		\hline
		DO 09:30-10:00 & Arbeiten im Team od. selbstständig \\
		\hline
		DO 10:00-10:20 & Pause \\
		\hline
		DO 10:20-12:00 & Arbeiten im Team od. selbstständig \\
		\hline
		FR 08:30 & Alle Mitglieder sind in der Teaminsel \\
		\hline
		FR 08:30-09:00 & Besprechung Team-intern erledigte Aufgaben \\
		& Fragen, weiteres Vorgehen \\
		\hline
		FR 09:00-10:00 & Arbeiten im Team od. selbständig \\
		\hline
		FR 10:00-10:20 & Pause \\
		\hline
		FR 10:20-11:00 & Arbeiten im Team od. selbstständig \\
		\hline
		FR 11:00-11:30 & Kurzbesprechung, Taskboard aktualisieren \\
		\hline
		FR 11:30-12:00 & Arbeiten im Team oder selbstständig (freiwillig)\\
		\hline
	\end{tabular}
	\caption{Wochenplan}
	\label{tab:Wochenplan}
\end{table}

\newpage

\section{Projektplan}
\label{sec:Projektplan}

\subsection{Prototypenplan Steuerung}
\label{ssec:ProtoSteuerung}
Da wir aufgrund interner Umstrukturierung in dieser späten Phase einen Grundsatzentscheid zur Mikrocontrollerplattform treffen mussten, ergeben sich fundamentale Änderungen. Vorher erarbeiteten Code ist daher nur bedingt brauchbar.

Es wurde daher in einer Taskforce aus Informatik und Elektrotechnik, in SCRUM angelehnten Sprints von einem Minimal Viable Prototype\footnote{MVP - Minimal Viable Product; Ein System mit Grundfunktionen welches iterativ erweitert wird, bis die gewünschte Funktionalität erreicht wird.} in vordefinierten Schritten unser Prototyp zu erarbeiten.

\vspace{1em}
\noindent
\begin{table}[h]
	\centering
	\begin{tabular}{|p{0.15\textwidth}|p{0.4\textwidth}|}
		\hline
		\textbf{Version} & \textbf{Funktionsumfang (Abb. \ref{fig:PrototypePlanung})} \\
		\hline
		v0.1 & 1,2,6\\
		\hline
		v0.2 & 1,2,3,6\\
		\hline
		v0.3 & 1,2,3,5,6\\
		\hline
		v1.0 & 1,2,3,4,5,6\\
		\hline
	\end{tabular}
	\caption{Prototypversionen}
	\label{tab:PrototypePlanung}
\end{table}

\begin{figure}[h!]
	\centering
	\includegraphics[width=0.5\linewidth,keepaspectratio]{PrototypePlanung}
	\caption{Funktionen Prototyp}
	\label{fig:PrototypePlanung}
\end{figure}

\subsection{Grober Rahmenplan}
\label{ssec:GrobRahmenplan}
An dieser Stelle ist der Projektplan grob umrissen, den Detaillierten finden Sie im Anhang.

\begin{figure}[h!]
	\includegraphics[width=\linewidth,keepaspectratio]{Rahmenplan}
	\caption{Grober Projektplan PREN02}
	\label{fig:GrobProjekt}
\end{figure}

\begin{table}[h!]
	\vspace{1em}
	\noindent
	\begin{tabular}{|p{0.2\textwidth}|p{0.2\textwidth}|p{0.55\textwidth}|}
		\hline
		\textbf{Meileinstein} & \textbf{Termin} & \textbf{Beschreibung} \\
		\hline
		Meilenstein 1 & 09.03.2018 12:00 Uhr & Detailplanung Projekt abgeschlossen, Entscheidung LoFive / Arduino gefallen\\
		\hline
		Meilenstein 2 & 19.04.2018 12:00 Uhr & Steuerungsprototyp v0.3 erreicht, Prototyp  montiert, Vorführung vor Experten\\
		\hline
		Meilenstein 3 & 25.05.2018 12:00 Uhr & Alle Komponenten freigegeben, voller Funktionsumfang erreicht\\
		\hline
		Meilenstein 4 & 03.07.2018 10:10 & Präsentation, Abgabe präsentation.ppt 16:00 auf Ilias\\
		\hline
		Meilenstein 5 & 04.07.2018 09:00 & Wettbewerb\\
		\hline
		Schlussabgabe PREN2 & 11.06.2018 16:00 & Schlussabgabe Dokumentation, 2 Exemplare (Nur Hauptteil), E309 \& Hauptteil und Anhang auf Ilias \\
		\hline
	\end{tabular}
	\caption{Meilensteine}
	\label{tab:Meilensteine}
\end{table}

\newpage

\section{Budgetplan}
\label{sec:Budgetplan}
Für den Bau der Teilfunktionsmuster in PREN1 dürfen maximal CHF 200.- ausgegeben werden. Die Tabelle \ref{tab:Budgetplan} dient daher vor allem der Kostenverfolgung, damit das Budget nicht überzogen wird.

\vspace{1em}
\noindent
\begin{table}[h!]
	\begin{tabular}{|p{0.3\textwidth}|p{0.15\textwidth}|p{0.225\textwidth}||p{0.225\textwidth}|}
		\hline
		\textbf{Artikel} & \textbf{Anzahl} & \textbf{Preis/Stk. CHF} & \textbf{Total CHF} \\
		\hline
		Rillenkugellager 608-2RS & 1 & 8.36 & 8.36 \\
		\hline
		Rundstab Buche & 3 & 0.9 & 2.7 \\
		\hline
		Buche Rundstäbe glatt & 1 & 0.7 & 0.7 \\
		\hline
		Aluminium Rohmaterial & 1 & 10 & 10 \\
		\hline
		Original Raspberry Pi Kamera Module V2 & 1 & 29.9 & 29.9 \\
		\hline
		Raspberry Pi Model B & 1 & 25.5 & 25.5 \\
		\hline
		Lofive & 1 & 25 & 25 \\
		\hline
		TURNIGY 1800MAH 3S 20C LIPO-PACK & 1 & 25 & 25\\
		\hline
		\textbf{Total} & & & \textbf{127.16} \\
		\hline
	\end{tabular}
	\caption{Budgetplan für das Projekt im PREN01}
	\label{tab:Budgetplan}
\end{table}

\section{Risikomanagement}
\label{ch:RisikoMgmt}
In diesem Kapitel werden mögliche Risiken während des Projektverlaufes aufgelistet. Dabei werden Projektrisiken nummeriert. Ihre Eintrittswahrscheinlichkeit und ihr Schadensausmass wird eingeschätzt. Besteht ein grosses Risiko, werden zusätzlich Massnahmen definiert.

\subsection{Definitionen}
\label{sec:Def}
\vspace{1em}
\noindent
Eintrittswahrscheinlichkeit:

\vspace{1em}
\noindent
\begin{tabular}{|p{0.06\textwidth}|p{0.2\textwidth}|p{0.7\textwidth}|}
	\hline
	\textbf{Stufe} & \textbf{Bezeichnung} & \textbf{Beschreibung} \\
	\hline
	1 & unvorstellbar & Möglich aber eher unwahrscheinlich. Tritt nie oder einmal in 14 Wochen auf \\
	\hline
	2 & unwahrscheinlich & Kann in 14 Wochen 1-5 Mal eintreten\\
	\hline
	3 & vorstellbar & Kann in 14 Wochen 6-8 Mal eintreten \\
	\hline
	4 & wahrscheinlich & Kann in 14 Wochen bis zu 10 Mal eintreten \\
	\hline
	5 & häufig & Kann in 14 Wochen 14 Mal eintreten\\
	\hline
\end{tabular}

\vspace{1em}
\noindent
Schadensausmass:

\vspace{1em}
\noindent
\begin{tabular}{|p{0.06\textwidth}|p{0.2\textwidth}|p{0.7\textwidth}|}
	\hline
	\textbf{Stufe} & \textbf{Bezeichnung} & \textbf{Beschreibung} \\
	\hline
	1 & unwesentlich & Die Aufgabenerfüllung wird höchstens geringfügig beeinträchtigt finanzieller Schaden ist im Rahmen des Projekts nicht beeinflussend. Personenschäden treten nicht auf \\
	\hline
	2 & geringfügig & Wahrnehmbare Gefährdung / Einfluss auf das Projekt. Personenschäden treten nicht auf \\
	\hline
	3 & mittelmässig & Wahrnehmbare Gefährdung / Einfluss auf das Projekt.Finanzieller Schaden strapaziert das Projektbudget
	Personenschäden treten nicht auf \\
	\hline
	4 & kritisch & Starke Gefährdung des Projekts. Finanzieller Schaden übersteigt das Projektbudget massiv. Personenschäden treten geringfügig auf \\
	\hline
	5 & katastrophal & Projektabbruch zur Folge. Finanzieller Schaden kann zum Projektstopp führen. Verletzung der Persönlichkeitsrechte
	\\
	\hline
\end{tabular}


\subsection{Risikokatalog}
\label{sec:Risikokatalog}
Legende:
\begin{itemize}
	\item \textbf{S}chadensausmass bei Eintreffen des Risikos
	\item \textbf{W}ahrscheinlichkeit das Risiko eintrifft
	\item \textbf{K}ategorie: \textbf{T}echnisches oder \textbf{P}rojektbezogenes Risiko
	\item \textbf{A}uswirkung auf das Projekt. Produkt aus S und W
\end{itemize}

\vspace{1em}
\noindent
\begin{tabular}{|p{0.03\textwidth}|p{0.75\textwidth}|p{0.03\textwidth}|p{0.03\textwidth}|p{0.03\textwidth}||p{0.03\textwidth}|}
	\hline
	\textbf{Nr.} & \textbf{Beschreibung / Risiko} & \textbf{K} & \textbf{S} & \textbf{W} & \textbf{A} \\
	\hline
	1 & Datenverlust & P & 5 & 1 & 5\\
	\hline
	2 & Zerstörung elektronischer Komponenten durch ESD & T & 3 & 3 & 9 \\
	\hline
	3 & Störung der Steuerungskomponenten durch Elektromotoren & T & 3 & 2 & 6 \\
	\hline
	4 & Fortbewegungsschwierigkeiten des Gerätes & T & 5 & 2 & 10 \\
	\hline
	5 & Schwankungen des Gerätes & T & 2 & 3 & 6 \\
	\hline
	6 & Probleme beim Transport der Last & T & 4 & 2 & 8 \\
	\hline
	7 & Probleme Fertigung oder Beschaffung Komponenten & P & 2 & 3 & 6 \\
	\hline
	8 & Stop über Zielplatzform schlägt fehl & T & 4 & 2 & 8 \\
	\hline
	9 & Fehlkommunikation im Team & P & 3 & 3 & 9 \\
	\hline
	10 & Teammitglied fällt aus & P & 3 & 2 & 6 \\
	\hline
	11 & Verzug bei Erstellung von Dokumenten & P & 3 & 2 & 6 \\
	\hline
	12 & Gerät startet nicht & T & 5 & 1 & 5 \\
	\hline
\end{tabular}

\subsection{Risikobewertung}
\label{sec:RisikoBewertung}
\begin{figure}[h!]
	\centering
	\includegraphics[width=.6\textwidth,keepaspectratio]{Risikomatrix}
	\caption{Die Risiken aus dem Risikokatalog graphisch dargestellt}
	\label{fig:Risikomatrix}
\end{figure}

\newpage
\subsection{Massnahmen}
\label{sec:Massnahmen}
\vspace{1em}
\noindent
\begin{table}[h!]
	\centering
	\begin{tabular}{|p{0.1\textwidth}|p{0.9\textwidth}|}
		\hline
		\textbf{Risiko Nr.} & \textbf{Massnahme} \\
		\hline
		1 & Es wird sowohl für Dokumentation und Programmcode mit GitHub als Versionskontrollsystem gearbeitet. Sonstige Daten befinden sich in der Dropbox. Dropbox wird als sicher eingestuft. \\
		\hline
		2 & Gesamtes Team ist im Umgang mit elektronischen Komponenten geschult. \\
		\hline
		3 & Magnetische Abschirmung des Elektromotors ist vorgesehen. Klare Aufteilung der Printplatte in Digital und Analog. \\
		\hline
		4 & Grösst mögliche Steigung des Seiles wird miteinbezogen. Laufkatze wird gegen rutschen und abstürzen gesichert. Akkuleistung wird genügend gross gewählt. \\
		\hline
		5 & Stabilisierung des Gerätes durch Gewicht oder Dämpfer. \\
		\hline
		6 & Erkennen, greifen und absetzen der Last muss durch Tests in 9 von 10 Mal funktionieren. \\
		\hline
		7 & Aufträge werden frühzeitig in Auftrag gegeben. Vor Auftragserteilung und Bestellung werden diese durch mindestens zwei Personen Überprüft.\\
		\hline
		8 & Zielerkennung wird unter verschiedenen Bedingungen (zum Beispiel Lichteinflüssen) getestet und muss in 9 von 10 Mal funktionieren. Laufkatze soll vor- und rückwärts fahren können. \\
		\hline
		9 & Bei Unklarheiten und Unwohlsein wird von jedem Teammitglied erwartet, dass er sich selbst meldet. Zur Veranschaulichung technischer Aspekte wird mit Skizzen gearbeitet. \\
		\hline
		10 & Aktueller Stand ist zu jedem Zeitpunkt allen klar. Aufgaben-Board ist immer aktuell. Im Falle eines Ausfalls kann so jemand anderes übernehmen. \\
		\hline
		11 & Mit Arbeiten und dazugehöriger Dokumentation wird frühzeitig begonnen. Der Aufwand wird grosszügig mit einberechnet / Puffer. \\
		\hline
		12 & Aufbau der Laufkatze wird geübt und soll nie länger als 1.5min dauern. Das Startsignal soll über zwei Verschiedenen Kanäle gesendet werden können. \\
		\hline
	\end{tabular}
	\caption{Massnahmen welche definiert wurden um die Risiken aus \ref{sec:Risikokatalog} zu minimieren}
	\label{tab:Massnahmen}
\end{table}

\newpage
\subsubsection{Effekt der Massnahmen}
\label{ssec:MassEffekt}
\begin{figure}[h!]
	\centering
	\begin{subfigure}[b]{0.8\textwidth}
		\includegraphics[keepaspectratio,width=\textwidth]{Risikomatrix}
	\end{subfigure}
	\begin{subfigure}[b]{0.8\textwidth}
		\includegraphics[keepaspectratio,width=\textwidth]{Risikomatrix_nachher}
	\end{subfigure}
	\caption{Gegenüberstellung der Risikomatrizen vor und nach den Massnahmen}
	\label{fig:Gegenueberstellung}
\end{figure}

\begin{figure}[h!]
	\centering
	\includegraphics[width=\textwidth,keepaspectratio]{Risikomatrix_Spinne}
	\caption{Die Risiken aus dem Risikokatalog graphisch dargestellt, die rote Linie zeigt die Risiken vor den Massnahmen, während die Grüne die Risiken nach dem Anwenden der Massnahmen darstellt}
	\label{fig:Risikomatrix_Spinne}
\end{figure}


\end{document}
